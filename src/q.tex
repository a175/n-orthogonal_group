% !TeX root =./x2.tex
% !TeX program = pdfpLaTeX

\chapter{線形代数の復習}

\section{抽象実ベクトル空間と線形写像}
和と実数によるスカラー倍が備わった空間を
実ベクトル空間と呼ぶ.
\begin{definition}
  $V$を集合とし,
  $+$, $\cdot$を二項演算
  \begin{align*}
    \bullet + \bullet \colon V\times V &\to V\\
    (v,w)\mapsto v+w\\
    \bullet \cdot \bullet \colon \RR\times V &\to V\\
    (a,w)\mapsto a\cdot w
  \end{align*}
  とし,
  $0_V$
  とする.
  以下の条件を満たすとき,
  $(V,+,\cdot,0_V)$を実ベクトル空間と呼ぶ:
  \begin{enumerate}
  \item $v,w,u\in V$に対し以下が成り立つ:
    \begin{enumerate}
    \item $(v+w)+u=v+(w+u)$.
    \item $v+w=w+v$.
    \item $v+0_V=v$.
    \item $v+(-1)\cdot v=0_V$.
    \end{enumerate}
  \item $v\in V$, $a,b\in\RR$に対し
    以下が成り立つ:
    \begin{enumerate}
    \item $(ab)\cdot v=a\cdot (b\cdot v)$.
    \item $1\cdot v=v$.
    \end{enumerate}      
  \item $v,w \in V$, $a,b\in\RR$に対し
    以下が成り立つ:
    \begin{enumerate}
    \item $(a+b)\cdot v=(a\cdot v) + (b\cdot v)$.
    \item $a\cdot (v+w)=(a\cdot v) + (a\cdot w)$.
    \end{enumerate}      
  \end{enumerate}
\end{definition}
\begin{remark}
  $(V,+,\cdot,0_V)$が実ベクトル空間であるとき,
  \begin{align*}
    0\cdot v
    &=(1-1)\cdot v\\
    &=1\cdot v + (-1)\cdot v\\
    &=v+ (-1)\cdot v= 0_V
  \end{align*}
  であるので,
  任意の$v\in V$に対し, $0v=0_V$である.
\end{remark}
\begin{remark}
  以下では, 文脈上誤解が生じないときには,
  $(V,+,\cdot,0_V)$が実ベクトル空間であることを単に
  $V$がベクトル空間であるといい,
  $0_V$のことを単に$0$と書いたりする.
  また, スカラー倍の際$\cdot$を省略し, $a\cdot v$を$av$の様に書くこともある.
\end{remark}
\begin{remark}
  $(-1)v$を$-v$, $w+(-v)$を$w-v$のように略記する.
\end{remark}


和とスカラー倍とコンパチブルな写像を線形写像と呼ぶ.
\begin{definition}
  $(V,+,\cdot,0_V)$, $(W,\oplus,\odot,0_W)$を実ベクトル空間とし,
  $f$を$V$から$W$への写像とする.
  以下の条件を満たすとき,
  $f$は$(V,+,\cdot,0_V)$から$(W,\oplus,\odot,0_W)$への($\RR$上の)線形写像であるという:
  \begin{enumerate}
  \item $v,v'\in V\implies f(v+v')=f(v)\oplus f(v')$.
  \item $a\in \RR$, $v\in V \implies f(a\cdot v)=a\odot f(v)$.
  \end{enumerate}
\end{definition}
\begin{remark}
  以下では, 文脈上誤解が生じないときには,
  $f$は$(V,+,\cdot,0_V)$から$(W,\oplus,\odot,0_W)$への($\RR$上の)線形写像であることを,
  単に$V$から$W$への線形写像であるという.
  また,
  本来であれば,
  線形写像$f\colon (V,+,\cdot,0_V)\to(W,\oplus,\odot,0_W)$
  などと書き表すべきであるが,
  線形写像$f\colon V\to W$などと書く.
\end{remark}
\begin{remark}
  $V$を実ベクトル空間とする.
  $V$上の恒等写像$\id_V$は線形写像である.
\end{remark}
\begin{definition}
  実ベクトル空間$V$, $W$に対し,
  \begin{align*}
    \Hom_\RR(V,W)
    =\Set{f\colon V\to W \text{: 線形}}
  \end{align*}
  とおく.
\end{definition}
\begin{remark}
  線形写像の合成は線形である.
  つまり,
  \begin{align*}
  f\in \Hom_\RR(V,W),\ 
  g\in \Hom_\RR(W,U)
  \implies
  g\circ f \in \Hom_\RR(W,U).
  \end{align*}
\end{remark}
\begin{definition}
  $V$, $W$をベクトル空間とし,
  $f\colon V\to W$を線形写像とする.
  次の条件を満たすとき,
  $f$は$V$から$W$への(線形)同型写像であるという:
  \begin{itemize}
  \item 次の条件を満たす線形写像$g\colon W\to V$が存在する:
    \begin{align*}
      g\circ f =\id_V,\\
      f\circ g =\id_W.
    \end{align*}
  \end{itemize}
  また, $V$から$W$への同型写像が存在するとき,
  $V$と$W$は(線形)同型であるといい$V\simeq W$と書く.
  つまり$V\simeq$と次は同値である:
  \begin{itemize}
  \item 次の条件を満たす線形写像$f\colon V\to W$と$g\colon W\to V$が存在する:
    \begin{align*}
      g\circ f =\id_V,\\
      f\circ g =\id_W.
    \end{align*}
  \end{itemize}
\end{definition}
\begin{remark}
  $V$と$W$が同型であるとき,
  $V$で成り立つことは同型写像$f$を通して$W$に翻訳できるので,
  $V$と$W$はベクトル空間として同じものであると思える.
\end{remark}
\begin{remark}
  線形写像$f\colon V\to W$が全単射であるとき,
  $f$の逆写像$f^{-1}$も線形である.
  したがって,
  $f\colon V\to W$が同型写像であることと,
  $f\colon V\to W$が全単射な線形写像であることは同値である.
\end{remark}
\begin{example}
  数ベクトルのなす集合
  \begin{align*}
    \RR^n=\Set{\begin{pmatrix}x_1\\\vdots\\x_n\end{pmatrix}|x_i\in\RR}
  \end{align*}
  は実ベクトル空間である. 
\end{example}
\begin{example}
  複素数のなす集合
  \begin{align*}
    \CC=\Set{x+y\sqrt{-1}|x,y\in\RR}
  \end{align*}
  は実ベクトル空間である.
  これは, 複素(数)平面と呼ばれる.
\end{example}
\begin{example}
  実数のなす集合$\RR=\RR^1$
  は実ベクトル空間である.
\end{example}

\begin{definition}
  $(V,+,\cdot,0_V)$,
  $(W,+,\cdot,0_W)$をベクトル空間とする.
  $f,g\in\Hom_\RR(V,W)$に対し,
  写像$f+g$を次で定義する:
  \begin{align*}
    f+g\colon V&\to W\\
    v&\mapsto f(v)+g(v).
  \end{align*}
  $a\in \RR$,
  $f\in\Hom_\RR(V,W)$に対し,
  写像$a\cdot f$を次で定義する:
  \begin{align*}
    a\cdot f\colon V&\to W\\
    v&\mapsto a\cdot f(v).
  \end{align*}
  また, 写像$\underline{0_W}$を次で定義する:
  \begin{align*}
    \underline{0_W}\colon V&\to W\\
    v&\mapsto 0_W.
  \end{align*}
\end{definition}
\begin{prop}
  ベクトル空間
  $(V,+,\cdot,0_V)$,
  $(W,+,\cdot,0_W)$に対し,
  $\underline{0_W}\in \Hom_\RR(V,W)$である.
  $f,g\in\Hom_\RR(V,W)$, $a\in \RR$に対し,
  $f+g, a\cdot f\in \Hom_\RR(V,W)$である.
  また, 
  $(\Hom_\RR(V,W),+,\cdot,\underline{0_W})$
  は実ベクトル空間である.
\end{prop}
\begin{remark}
  \label{A:rem:linearmap:sum:sc}
  $\Hom_\RR(V,W)$がベクトル空間であるので,
  和とスカラー倍は分配則などを満たす.
  さらに, 合成とは以下を満たす:
  $f,f'\in\Hom_\RR(V,W)$,
  $g,g'\in\Hom_\RR(W,U)$,
  $a\in\RR$に対し,
  \begin{enumerate}
  \item $(g+g')\circ f=g\circ f+g'\circ f$,
  \item $g\circ (f+f')=g\circ f+g\circ f'$,
  \item $c(g\circ f)=(cg)\circ f=g\circ(cf)$.    
  \end{enumerate}
\end{remark}

\begin{definition}
  $V$をベクトル空間とする.
  \begin{align*}
    V^\ast &= \Hom_\RR(V,\RR)\\
    &=\Set{\varphi\colon V\to \RR\text{: 線形写像}}
  \end{align*}
  とおき, ベクトル空間$V^\ast$を$V$の双対空間と呼ぶ.
\end{definition}


$V$, $W$をベクトル空間とし,
$f$を$V$から$W$への線形写像とする.
$\varphi\in W^\ast$とすると,
$\varphi$は$W$から$\RR$への線形写像であるので,
\begin{align*}
  \varphi\circ f \colon V&\to \RR\\
  v&\mapsto \varphi(f(v))
\end{align*}
は線形である.
つまり, $\varphi\circ f \in \Hom_\RR(V,\RR)=V^\ast$である.
よって,
写像
\begin{align*}
  \Phi_f\colon W^\ast &\to V^\ast\\
  \varphi &\mapsto \varphi\circ f
\end{align*}
が定義できる.
\begin{lemma}
$V$, $W$をベクトル空間とし,
$f$を$V$から$W$への線形写像とし,
\begin{align*}
  \Phi_f\colon W^\ast &\to V^\ast\\
  \varphi &\mapsto \varphi\circ f
\end{align*}
とすると, $\Phi_f\in \Hom_\RR(W^\ast,V^\ast)$.
\end{lemma}
\begin{proof}
  $\varphi,\psi\in W^\ast$に対し,
  \begin{align*}
    \Phi_f(\varphi+\psi)&=(\varphi+\psi)\circ f \in W^\ast\\
    \Phi_f(\varphi)+\Phi_f(\psi)&=\varphi\circ f+\psi\circ f \in W^\ast
  \end{align*}
  である.
  $v\in V$に対し,
  \begin{align*}
    (\Phi_f(\varphi+\psi))(v)
    &=((\varphi+\psi)\circ f )(v)\\
    &=(\varphi+\psi)(f (v))\\
    &=\varphi(f (v))+\psi(f (v))\\
    (\Phi_f(\varphi)+\Phi_f(\psi))(v)
    &=(\varphi\circ f+\psi\circ f )(v)\\
    &=(\varphi\circ f)(v)+(\psi\circ f )(v)\\
    &=\varphi(f(v))+\psi( f (v))
  \end{align*}
  となるので%
  \footnote{\Cref{A:rem:linearmap:sum:sc}において証明せずに紹介した事実の証明をここで与えている.}%
  ,
  $\Phi_f(\varphi+\psi)=\Phi_f(\varphi)+\Phi_f(\psi)$.

  $a\in \RR$,
  $\varphi\in W^\ast$に対し,
  \begin{align*}
    \Phi_f(a\varphi)&=(a\varphi)\circ f \in W^\ast\\
    a\Phi_f(\varphi)&=a(\varphi\circ f)\in W^\ast
  \end{align*}
  である.
  $v\in V$に対し,
  \begin{align*}
    (\Phi_f(a\varphi))(v)
    &=((a\varphi)\circ f)(v)\\ 
    &=(a\varphi)(f(v))\\ 
    &=a(\varphi(f(v))\\ 
    (a\Phi_f(\varphi))(v)
    &=(a(\varphi\circ f))(v)\\
    &=a((\varphi\circ f)(v))\\
    &=a(\varphi(f(v)))
  \end{align*}
  となるので%
  \footnote{\Cref{A:rem:linearmap:sum:sc}において証明せずに紹介した事実の証明をここで与えている.}%
  ,
  $\Phi_f(a\varphi)=a\Phi_f(\varphi)$.
\end{proof}
\begin{definition}
  $V,W$をベクトル空間とする.
  $f\in \Hom(V,W)$に対し,
  $\transposed{f}\in \Hom(W^\ast,V^\ast)$を,
  \begin{align*}
    \transposed{f}\colon W^\ast &\to V^\ast\\
    \varphi &\mapsto \varphi\circ f
  \end{align*}
  で定義し, $f$の転置と呼ぶ.
\end{definition}

(余談)
\begin{align*}
  \Phi\colon \Hom_\RR(V,W)&\to \Hom(W^\ast,V^\ast)  \\
  f&\mapsto \Phi_f=\transposed{f}
\end{align*}
は線形である.
つまり,
\begin{align*}
  \Phi\in\Hom_\RR(\Hom_\RR(V,W),\Hom_\RR(W^\ast,V^ast)).
\end{align*}
\begin{proof}
  $f,f'\in \Hom_\RR(V,W)$とする.
  $\Phi_{f+f'},\Phi_{f}+\Phi_{f'}\in\Hom_\RR(W^\ast,V^\ast)$
  である.
  $\varphi\in W^\ast$に対し,
  \begin{align*}
    \Phi_{f+f'}(\varphi)
    &=(\transposed{(f+f')})(\varphi)\\
    &=\varphi\circ(f+f')\in V^\ast,\\
    (\Phi_{f}+\Phi_{f'})(\varphi)
    &=(\transposed{f}+\transposed{f'})(\varphi)\\
    &=(\transposed{f})(\varphi)+(\transposed{f'})(\varphi)\\
    &=\varphi\circ f+\varphi\circ f'\in V^\ast.
  \end{align*}
  $v\in V$に対し,
  \begin{align*}
    (\Phi_{f+f'}(\varphi))(v)
    &=(\varphi\circ(f+f'))(v)\\
    &=\varphi((f+f')(v))\\
    &=\varphi(f(v)+f'(v))\\
    &=\varphi(f(v))+\varphi(f'(v)),\\
    ((\Phi_{f}+\Phi_{f'})(\varphi))(v)
    &=(\varphi\circ f+\varphi\circ f')(v)\\
    &=(\varphi\circ f)(v)+(\varphi\circ f')(v)\\
    &=\varphi(f(v))+\varphi(f'(v)).
  \end{align*}
  よって, $\Phi_{f+f'}=\Phi_{f}+\Phi_{f'}$.
 
  $a\in\RR$,
  $f\in \Hom_\RR(V,W)$とする.
  $\Phi_{af},\Phi_{f}+\Phi_{f'}\in\Hom_\RR(W^\ast,V^\ast)$
  である.
  $\varphi\in W^\ast$に対し,
  \begin{align*}
    \Phi_{af}(\varphi)&=(\transposed{(a f)})(\varphi)=\varphi\circ(af)\in V^\ast\\
    (a\Phi_{f})(\varphi)&=a(\transposed{f}(\varphi))=a(\varphi\circ f)\in V^\ast.
  \end{align*}
  $v\in V$に対し,
  \begin{align*}
    (\Phi_{af}(\varphi))(v)
    &=(\varphi\circ(af)) (v)\\
    &=\varphi(af(v))\\
    &=a\varphi(f(v))\\
    ((a\Phi_{f})(\varphi))(v)
    &=(a(\varphi\circ f))(v)\\
    &=a(\varphi\circ f)(v))\\
    &=a(\varphi(f(v))).
  \end{align*}
  よって, $\Phi_{af}=a\Phi_{f}$.
\end{proof}


\section{実内積と等長写像}
