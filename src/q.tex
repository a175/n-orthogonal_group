% !TeX root =./x2.tex
% !TeX program = pdfpLaTeX

\chapter{線形代数の復習}
ここでは, ベクトル空間と内積についての事実を復習する.
証明は他の文献に譲る.
また, 復習とは書いたものの,
通常とは異なるまとめ方になっているので,
わかりにくい.
ここの内容を読まずとも本文は読めるし,
ここの内容を理解したからと言って本文が読みやすくなるわけでもない.
基本的には本文とは無関係である.
また, ここでの記号の使い方は本文とは異なる.


\section{抽象実ベクトル空間と線形写像}
抽象実ベクトル空間や線型写像について
まとめている.
線型写像$f\colon V\to W$の転置について,
$W^\ast\to V^\ast$としてcanonicalに得られるもの
(ここでは$\transposed{f}$と書いている)
と,
内積を決めることにより
$W\to V$として得られるもの
(ここでは$f^\top$と書いている)
についてまとめた.
\subsection{双対空間と線型写像の転置}
和と実数によるスカラー倍が備わった空間を
実ベクトル空間と呼ぶ.
\begin{definition}
  $V$を集合とし,
  $+$, $\cdot$を二項演算
  \begin{align*}
    \bullet + \bullet \colon V\times V &\to V\\
    (v,w)&\mapsto v+w\\
    \bullet \cdot \bullet \colon \RR\times V &\to V\\
    (a,w)&\mapsto a\cdot w
  \end{align*}
  とし,
  $0_V$
  とする.
  以下の条件を満たすとき,
  $(V,+,\cdot,0_V)$を実ベクトル空間と呼ぶ:
  \begin{enumerate}
  \item $v,w,u\in V$に対し以下が成り立つ:
    \begin{enumerate}
    \item $(v+w)+u=v+(w+u)$.
    \item $v+w=w+v$.
    \item $v+0_V=v$.
    \item $v+(-1)\cdot v=0_V$.
    \end{enumerate}
  \item $v\in V$, $a,b\in\RR$に対し
    以下が成り立つ:
    \begin{enumerate}
    \item $(ab)\cdot v=a\cdot (b\cdot v)$.
    \item $1\cdot v=v$.
    \end{enumerate}      
  \item $v,w \in V$, $a,b\in\RR$に対し
    以下が成り立つ:
    \begin{enumerate}
    \item $(a+b)\cdot v=(a\cdot v) + (b\cdot v)$.
    \item $a\cdot (v+w)=(a\cdot v) + (a\cdot w)$.
    \end{enumerate}      
  \end{enumerate}
\end{definition}
\begin{remark}
  $(V,+,\cdot,0_V)$が実ベクトル空間であるとき,
  \begin{align*}
    0\cdot v
    &=(1-1)\cdot v\\
    &=1\cdot v + (-1)\cdot v\\
    &=v+ (-1)\cdot v= 0_V
  \end{align*}
  であるので,
  任意の$v\in V$に対し, $0v=0_V$である.
\end{remark}
\begin{remark}
  以下では, 文脈上誤解が生じないときには,
  $(V,+,\cdot,0_V)$が実ベクトル空間であることを単に
  $V$がベクトル空間であるといい,
  $0_V$のことを単に$0$と書いたりする.
  また, スカラー倍の際$\cdot$を省略し, $a\cdot v$を$av$の様に書くこともある.
\end{remark}
\begin{remark}
  $(-1)v$を$-v$, $w+(-v)$を$w-v$のように略記する.
\end{remark}


和とスカラー倍とコンパチブルな写像を線形写像と呼ぶ.
\begin{definition}
  $(V,+,\cdot,0_V)$, $(W,\oplus,\odot,0_W)$を実ベクトル空間とし,
  $f$を$V$から$W$への写像とする.
  以下の条件を満たすとき,
  $f$は$(V,+,\cdot,0_V)$から$(W,\oplus,\odot,0_W)$への($\RR$上の)線形写像であるという:
  \begin{enumerate}
  \item $v,v'\in V\implies f(v+v')=f(v)\oplus f(v')$.
  \item $a\in \RR$, $v\in V \implies f(a\cdot v)=a\odot f(v)$.
  \end{enumerate}
\end{definition}
\begin{remark}
  以下では, 文脈上誤解が生じないときには,
  $f$は$(V,+,\cdot,0_V)$から$(W,\oplus,\odot,0_W)$への($\RR$上の)線形写像であることを,
  単に$V$から$W$への線形写像であるという.
  また,
  本来であれば,
  線形写像$f\colon (V,+,\cdot,0_V)\to(W,\oplus,\odot,0_W)$
  などと書き表すべきであるが,
  線形写像$f\colon V\to W$などと書く.
\end{remark}
\begin{remark}
  $V$を実ベクトル空間とする.
  $V$上の恒等写像$\id_V$は線形写像である.
\end{remark}
\begin{definition}
  実ベクトル空間$V$, $W$に対し,
  \begin{align*}
    \Hom_\RR(V,W)
    =\Set{f\colon V\to W \text{: 線形}}
  \end{align*}
  とおく.
\end{definition}
\begin{remark}
  線形写像の合成は線形である.
  つまり,
  \begin{align*}
  f\in \Hom_\RR(V,W),\ 
  g\in \Hom_\RR(W,U)
  \implies
  g\circ f \in \Hom_\RR(W,U).
  \end{align*}
\end{remark}
\begin{definition}
  $V$, $W$をベクトル空間とし,
  $f\colon V\to W$を線形写像とする.
  次の条件を満たすとき,
  $f$は$V$から$W$への(線形)同型写像であるという:
  \begin{itemize}
  \item 次の条件を満たす線形写像$g\colon W\to V$が存在する:
    \begin{align*}
      g\circ f &=\id_V,\\
      f\circ g &=\id_W.
    \end{align*}
  \end{itemize}
  また, $V$から$W$への同型写像が存在するとき,
  $V$と$W$は(線形)同型であるといい$V\simeq W$と書く.
  つまり$V\simeq$と次は同値である:
  \begin{itemize}
  \item 次の条件を満たす線形写像$f\colon V\to W$と$g\colon W\to V$が存在する:
    \begin{align*}
      g\circ f &=\id_V,\\
      f\circ g &=\id_W.
    \end{align*}
  \end{itemize}
\end{definition}
\begin{remark}
  $V$と$W$が同型であるとき,
  $V$で成り立つことは同型写像$f$を通して$W$に翻訳できるので,
  $V$と$W$はベクトル空間として同じものであると思える.
\end{remark}
\begin{remark}
  線形写像$f\colon V\to W$が全単射であるとき,
  $f$の逆写像$f^{-1}$も線形である.
  したがって,
  $f\colon V\to W$が同型写像であることと,
  $f\colon V\to W$が全単射な線形写像であることは同値である.
\end{remark}
\begin{example}
  数ベクトルのなす集合
  \begin{align*}
    \RR^n=\Set{\begin{pmatrix}x_1\\\vdots\\x_n\end{pmatrix}|x_i\in\RR}
  \end{align*}
  は実ベクトル空間である. 
\end{example}
\begin{example}
  複素数のなす集合
  \begin{align*}
    \CC=\Set{x+y\sqrt{-1}|x,y\in\RR}
  \end{align*}
  は実ベクトル空間である.
  これは, 複素(数)平面と呼ばれる.
\end{example}
\begin{example}
  実数のなす集合$\RR=\RR^1$
  は実ベクトル空間である.
\end{example}

\begin{definition}
  $(V,+,\cdot,0_V)$,
  $(W,+,\cdot,0_W)$をベクトル空間とする.
  $f,g\in\Hom_\RR(V,W)$に対し,
  写像$f+g$を次で定義する:
  \begin{align*}
    f+g\colon V&\to W\\
    v&\mapsto f(v)+g(v).
  \end{align*}
  $a\in \RR$,
  $f\in\Hom_\RR(V,W)$に対し,
  写像$a\cdot f$を次で定義する:
  \begin{align*}
    a\cdot f\colon V&\to W\\
    v&\mapsto a\cdot f(v).
  \end{align*}
  また, 写像$\underline{0_W}$を次で定義する:
  \begin{align*}
    \underline{0_W}\colon V&\to W\\
    v&\mapsto 0_W.
  \end{align*}
\end{definition}
\begin{prop}
  ベクトル空間
  $(V,+,\cdot,0_V)$,
  $(W,+,\cdot,0_W)$に対し,
  $\underline{0_W}\in \Hom_\RR(V,W)$である.
  $f,g\in\Hom_\RR(V,W)$, $a\in \RR$に対し,
  $f+g, a\cdot f\in \Hom_\RR(V,W)$である.
  また, 
  $(\Hom_\RR(V,W),+,\cdot,\underline{0_W})$
  は実ベクトル空間である.
\end{prop}
\begin{remark}
  \label{A:rem:linearmap:sum:sc}
  $\Hom_\RR(V,W)$がベクトル空間であるので,
  和とスカラー倍は分配則などを満たす.
  さらに, 合成とは以下を満たす:
  $f,f'\in\Hom_\RR(V,W)$,
  $g,g'\in\Hom_\RR(W,U)$,
  $a\in\RR$に対し,
  \begin{enumerate}
  \item $(g+g')\circ f=g\circ f+g'\circ f$,
  \item $g\circ (f+f')=g\circ f+g\circ f'$,
  \item $c(g\circ f)=(cg)\circ f=g\circ(cf)$.    
  \end{enumerate}
\end{remark}

\begin{definition}
  $V$をベクトル空間とする.
  \begin{align*}
    V^\ast &= \Hom_\RR(V,\RR)\\
    &=\Set{\varphi\colon V\to \RR\text{: 線形写像}}
  \end{align*}
  とおき, ベクトル空間$V^\ast$を$V$の双対空間と呼ぶ.
\end{definition}


$V$, $W$をベクトル空間とし,
$f$を$V$から$W$への線形写像とする.
$\varphi\in W^\ast$とすると,
$\varphi$は$W$から$\RR$への線形写像であるので,
\begin{align*}
  \varphi\circ f \colon V&\to \RR\\
  v&\mapsto \varphi(f(v))
\end{align*}
は線形である.
つまり, $\varphi\circ f \in \Hom_\RR(V,\RR)=V^\ast$である.
よって,
写像
\begin{align*}
  \Phi_f\colon W^\ast &\to V^\ast\\
  \varphi &\mapsto \varphi\circ f
\end{align*}
が定義できる.
\begin{lemma}
$V$, $W$をベクトル空間とし,
$f$を$V$から$W$への線形写像とし,
\begin{align*}
  \Phi_f\colon W^\ast &\to V^\ast\\
  \varphi &\mapsto \varphi\circ f
\end{align*}
とすると, $\Phi_f\in \Hom_\RR(W^\ast,V^\ast)$.
\end{lemma}
\begin{proof}
  $\varphi,\psi\in W^\ast$に対し,
  \begin{align*}
    \Phi_f(\varphi+\psi)&=(\varphi+\psi)\circ f \in W^\ast\\
    \Phi_f(\varphi)+\Phi_f(\psi)&=\varphi\circ f+\psi\circ f \in W^\ast
  \end{align*}
  である.
  $v\in V$に対し,
  \begin{align*}
    (\Phi_f(\varphi+\psi))(v)
    &=((\varphi+\psi)\circ f )(v)\\
    &=(\varphi+\psi)(f (v))\\
    &=\varphi(f (v))+\psi(f (v))\\
    (\Phi_f(\varphi)+\Phi_f(\psi))(v)
    &=(\varphi\circ f+\psi\circ f )(v)\\
    &=(\varphi\circ f)(v)+(\psi\circ f )(v)\\
    &=\varphi(f(v))+\psi( f (v))
  \end{align*}
  となるので%
  \footnote{\Cref{A:rem:linearmap:sum:sc}において証明せずに紹介した事実の証明をここで与えている.}%
  ,
  $\Phi_f(\varphi+\psi)=\Phi_f(\varphi)+\Phi_f(\psi)$.

  $a\in \RR$,
  $\varphi\in W^\ast$に対し,
  \begin{align*}
    \Phi_f(a\varphi)&=(a\varphi)\circ f \in W^\ast\\
    a\Phi_f(\varphi)&=a(\varphi\circ f)\in W^\ast
  \end{align*}
  である.
  $v\in V$に対し,
  \begin{align*}
    (\Phi_f(a\varphi))(v)
    &=((a\varphi)\circ f)(v)\\ 
    &=(a\varphi)(f(v))\\ 
    &=a(\varphi(f(v))\\ 
    (a\Phi_f(\varphi))(v)
    &=(a(\varphi\circ f))(v)\\
    &=a((\varphi\circ f)(v))\\
    &=a(\varphi(f(v)))
  \end{align*}
  となるので%
  \footnote{\Cref{A:rem:linearmap:sum:sc}において証明せずに紹介した事実の証明をここで与えている.}%
  ,
  $\Phi_f(a\varphi)=a\Phi_f(\varphi)$.
\end{proof}
\begin{definition}
  $V,W$をベクトル空間とする.
  $f\in \Hom_\RR(V,W)$に対し,
  $\transposed{f}\in \Hom_\RR(W^\ast,V^\ast)$を,
  \begin{align*}
    \transposed{f}\colon W^\ast &\to V^\ast\\
    \varphi &\mapsto \varphi\circ f
  \end{align*}
  で定義し, $f$の転置と呼ぶ.
\end{definition}

(余談)
\begin{align*}
  \Phi\colon \Hom_\RR(V,W)&\to \Hom_\RR(W^\ast,V^\ast)  \\
  f&\mapsto \Phi_f=\transposed{f}
\end{align*}
は線形である.
つまり,
\begin{align*}
  \Phi\in\Hom_\RR(\Hom_\RR(V,W),\Hom_\RR(W^\ast,V^ast)).
\end{align*}
\begin{proof}
  $f,f'\in \Hom_\RR(V,W)$とする.
  $\Phi_{f+f'},\Phi_{f}+\Phi_{f'}\in\Hom_\RR(W^\ast,V^\ast)$
  である.
  $\varphi\in W^\ast$に対し,
  \begin{align*}
    \Phi_{f+f'}(\varphi)
    &=(\transposed{(f+f')})(\varphi)\\
    &=\varphi\circ(f+f')\in V^\ast,\\
    (\Phi_{f}+\Phi_{f'})(\varphi)
    &=(\transposed{f}+\transposed{f'})(\varphi)\\
    &=(\transposed{f})(\varphi)+(\transposed{f'})(\varphi)\\
    &=\varphi\circ f+\varphi\circ f'\in V^\ast.
  \end{align*}
  $v\in V$に対し,
  \begin{align*}
    (\Phi_{f+f'}(\varphi))(v)
    &=(\varphi\circ(f+f'))(v)\\
    &=\varphi((f+f')(v))\\
    &=\varphi(f(v)+f'(v))\\
    &=\varphi(f(v))+\varphi(f'(v)),\\
    ((\Phi_{f}+\Phi_{f'})(\varphi))(v)
    &=(\varphi\circ f+\varphi\circ f')(v)\\
    &=(\varphi\circ f)(v)+(\varphi\circ f')(v)\\
    &=\varphi(f(v))+\varphi(f'(v)).
  \end{align*}
  よって, $\Phi_{f+f'}=\Phi_{f}+\Phi_{f'}$.
 
  $a\in\RR$,
  $f\in \Hom_\RR(V,W)$とする.
  $\Phi_{af},\Phi_{f}+\Phi_{f'}\in\Hom_\RR(W^\ast,V^\ast)$
  である.
  $\varphi\in W^\ast$に対し,
  \begin{align*}
    \Phi_{af}(\varphi)&=(\transposed{(a f)})(\varphi)=\varphi\circ(af)\in V^\ast\\
    (a\Phi_{f})(\varphi)&=a(\transposed{f}(\varphi))=a(\varphi\circ f)\in V^\ast.
  \end{align*}
  $v\in V$に対し,
  \begin{align*}
    (\Phi_{af}(\varphi))(v)
    &=(\varphi\circ(af)) (v)\\
    &=\varphi(af(v))\\
    &=a\varphi(f(v))\\
    ((a\Phi_{f})(\varphi))(v)
    &=(a(\varphi\circ f))(v)\\
    &=a(\varphi\circ f)(v))\\
    &=a(\varphi(f(v))).
  \end{align*}
  よって, $\Phi_{af}=a\Phi_{f}$.
\end{proof}


\subsection{内積空間と線型写像の転置}
\begin{definition}
  $V$を実ベクトル空間とする.
  \begin{align*}
    \Braket{\bullet,\bullet}_V
    \colon
    V\times V \to \RR\\
    (v,w) \mapsto\Braket{v,w}_V
  \end{align*}
  が以下の条件を満たすとき,
  $(V,\Braket{\bullet,\bullet}_V)$は(実)内積空間であるという:
  \begin{enumerate}
  \item
    \label{def:inner:item:line}
    $a\in\RR$, $v,w,u\in V\implies$
    \begin{align}
      \Braket{v+w,u}_V&=\Braket{v,u}_V+\Braket{w,u}_V,
      \label{def:inner:item:line:1:sum}
      \\
      \Braket{av,u}_V&=a\Braket{v,u}_V.
      \label{def:inner:item:line:1:sca}
    \end{align}
  \item
    \label{def:inner:item:sym}
    $v,w\in V \implies \Braket{v,w}_V=\Braket{w,v}_V$.
  \item 
    \label{def:inner:item:nondege}
    $v\in V$に対し次が成り立つ:
    \begin{align}
      v=0_V
      \iff
      \forall w\in V,\Braket{v,w}_V=0
      \label{def:inner:item:nondege:1}
    \end{align}
  \item
    \label{def:inner:item:semi}
    $v\in V \implies \Braket{v,v}_V\geq 0$.
  \item
    \label{def:inner:item:posi}
    $\Braket{v,v}_V=0\iff v=0_V$.
  \end{enumerate}
\end{definition}
\begin{remark}
  文脈上誤解がない場合には,
  $(V,\Braket{\bullet,\bullet}_V)$が内積空間であることを,
  単に$V$が内積空間であると言うこともある.
  $\Braket{\bullet,\bullet}_V$を$V$の内積と呼ぶ.
\end{remark}
\begin{remark}
  \Cref{def:inner:item:sym}を満たすことを
  $\Braket{\bullet,\bullet}_V$が対称であるという.
\end{remark}
\begin{remark}
  \Cref{def:inner:item:line}
  を満たすことを
  $\Braket{\bullet,\bullet}_V$が
  第一成分に関して線形
  であるという.
  \Cref{def:inner:item:sym,def:inner:item:line}から
  \begin{align}
    \Braket{u,v+w}_V&=\Braket{u,v}_V+\Braket{u,w}_V,
    \label{def:inner:item:line:2:sum}
    \\
    \Braket{u,av}_V&=a\Braket{u,v}_V    
    \label{def:inner:item:line:2:sca}
  \end{align}
  がわかる.
  \Cref{def:inner:item:line:2:sum,def:inner:item:line:2:sca}
  を満たすことを,
  $\Braket{\bullet,\bullet}_V$が
  第二成分に関して線形
  であるという.
  \Cref{def:inner:item:line:2:sum,def:inner:item:line:2:sca,def:inner:item:line:1:sum,def:inner:item:line:1:sca}
  を満たすことを,
  $\Braket{\bullet,\bullet}_V$が
  双線形
  であるという.
\end{remark}
\begin{remark}
  \Cref{def:inner:item:sym,def:inner:item:nondege:1}
  から,
  \begin{align}
      v=0_V
      \iff
      \forall w\in V,\Braket{w,v}_V=0
      \label{def:inner:item:nondege:2}
  \end{align}
  がわかる.
  \Cref{def:inner:item:nondege:1,def:inner:item:nondege:2}
  が成り立つことを,
  $\Braket{\bullet,\bullet}_V$が
  非退化
  であるという.

  \Cref{def:inner:item:nondege:1}は対偶をを取ると,
  \begin{align*}
    v\in V\setminus\Set{0_V}
    \iff
    \exists w\in V \text{ such that }
    \Braket{v,w}_V\neq 0
  \end{align*}
  となるのでこれが条件として書かれることもある.
  また,
  \Cref{def:inner:item:line:1:sca}
  から,
  \begin{align*}
    v=0_V
    \implies
    \forall w\in V,\Braket{v,w}_V=0
  \end{align*}
  が成り立つので,
  \begin{align}
    v\in V\setminus\Set{0_V}
    \implies
    \exists w\in V \text{ such that }
    \Braket{v,w}_V\neq 0
    \label{def:inner:item:nondege:1x}
  \end{align}
  のみが条件として書かれることもある.


  
  \Cref{def:inner:item:line}
    から, $\Braket{v_0,v}_V= \Braket{v_1,v}_V$ならば,
  \begin{align*}
    0=\Braket{v_0,v}_V-\Braket{v_1,v}_V
    =\Braket{v_0-v_1,v}_V
  \end{align*}
  となる.
  したがって,
  任意の$v\in V$に対して
  $\Braket{v_0,v}_V= \Braket{v_1,v}_V$ならば,
  \Cref{def:inner:item:nondege:1}から,
  $v_0-v_1=0_V$となり, $v_0=v_1$であることがわかる.
  逆は明らかであるから,
  \Cref{def:inner:item:line,def:inner:item:nondege:1}
  から
  \begin{align*}
    v_0=v_1\iff \forall v\in V, \Braket{v,v_0}_V= \Braket{v,v_1}_V
  \end{align*}
  がわかる.
\end{remark}
\begin{remark}
\Cref{def:inner:item:semi}を満たすことを,
  $\Braket{\bullet,\bullet}_V$は
半正定値(positive semidefinite)
  であるという.
  \Cref{def:inner:item:posi,def:inner:item:semi}を満たすことを,
  $\Braket{\bullet,\bullet}_V$は
  正定値(positive definite)
  であるという.

  \Cref{def:inner:item:line:1:sca}から,
  \begin{align*}
    v=0_V\implies \Braket{v,v}_V=0 
  \end{align*}
  はわかるので,
  \Cref{def:inner:item:posi}
  の条件の代わりに
  \begin{align*}
    \Braket{v,v}_V=0\implies v=0_V.
  \end{align*}
  と書かれることもある.
  この条件の対偶は,
  \begin{align*}
    v\in V\setminus\Set{0_V}
    \implies
    \Braket{v,v}_V\neq 0
  \end{align*}
  であるが,
  \Cref{def:inner:item:semi}を満たすときには,  
  \begin{align*}
    v\in V\setminus\Set{0_V}
    \implies
    \Braket{v,v}_V> 0
  \end{align*}
  と書くことができる.
  
  \Cref{def:inner:item:posi}を認めると,
  \Cref{def:inner:item:nondege:1x}の$w$として$v$自身がとれるので,
  \Cref{def:inner:item:nondege}
  が導かれる.
  そのため
  \Cref{def:inner:item:nondege}
  が条件として省かれることも多い.
\end{remark}

\begin{remark}
  内積とは,
  非退化正定値対称双線型形式のこと
  (形式というのは写像$V\times V\to \RR$を意味している).
\end{remark}

\begin{remark}
  複素ベクトルのときには,
  $\Braket{\bullet,\bullet}_V\colon V\times V\to \CC$
  として,
  \Cref{def:inner:item:sym}の条件を,
  \begin{align*}
    v,w\in V \implies \Braket{v,w}_V=\overline{\Braket{w,v}_V}
  \end{align*}
  に変更したものを考える.
  ただし$\overline{\bullet}$は複素共軛を表す.
  この条件を満たすことを,
  $\Braket{\bullet,\bullet}_V$
  はエルミート対称であるという.
\end{remark}

内積空間の構造とコンパチブルな写像を等長写像と呼ぶ.
\begin{definition}
  $(V,\Braket{}_V)$, $(W,\Braket{}_W)$を内積空間とし,
  $f\colon V\to W$を線型写像とする.
  以下の条件を満たすとき,
  $f$は等長写像であるという.
  \begin{align*}
    v,v'\in V\implies
    \Braket{v,v'}_V
    =
    \Braket{f(v),f(v')}_W
  \end{align*}
\end{definition}

\begin{remark}
  以下では, 文脈上誤解が生じないときには,
  $f$は
  $(V,\Braket{}_V)$から$(W,\Braket{}_W)$への
  等長写像
  であることを,
  単に$V$から$W$への等長写像であるという.
  また,
  本来であれば,
  等長写像$f\colon (V,\Braket{}_V) \to (W,\Braket{}_W)$
  などと書き表すべきであるが,
  等長写像$f\colon V\to W$などと書く.
\end{remark}

\begin{definition}
  $V$, $W$を内積空間とする.
  線型同型写像かつ等長写像である$f\colon V\to W$が存在するとき,
  $V$と$W$は内積空間として同型であるという.
\end{definition}
\begin{remark}
  $V$と$W$が内積空間として同型であるとき,
  $V$と$W$は内積空間として同じものであると思える.
\end{remark}
\begin{remark}
  等長写像の合成は等長写像である.
\end{remark}
\begin{remark}
  $V$を内積空間とする.
  $v\in V$に対し,
  \begin{align*}
    \|v\|_V = \sqrt{\Braket{v,v}_V}
  \end{align*}
  とおき, $\|\bullet\|_V$をノルムと呼ぶ.
  $\|v\|_V$は$v$の`長さ'に相当する.
  $f\colon V\to W$を等長写像とすると,
  \begin{align*}
    \|v\|_V &= \sqrt{\Braket{v,v}_V}\\
    &= \sqrt{\Braket{f(v),f(v)}_W}\\
    &= \|f(v)\|_W
  \end{align*}
  となり, ノルムが$f$によって変化しない.
\end{remark}
\begin{remark}
  $V$, $W$を内積空間とし,
  $f\colon V\to W$を等長写像とする.
  $f(v)=0_W$とする.
  このとき,
  $\|v\|_V=\|f(v)\|_W=\|0_W\|=0$
  であるので, $v=0_V$であることがわかる.
  したがって,
  $f$は単射である.
\end{remark}
\begin{definition}
  $(V,\Braket{}_V)$を内積空間とする.
  $O(V,\Braket{}_V)=\Set{f\in\Hom_\RR(V,V)\text{: 等長}}$
  とおく.
  $O(V)$を$V$上の直交群と呼ぶ.
  また,
  $f\in O(V)$を$V$上の直交変換と呼ぶ.
\end{definition}


  $V$を内積空間であるとする.
  $v_0\in V$に対し,
  \begin{align*}
    \beta_{v_0}\colon
    V&\to \RR\\
    v&\mapsto \Braket{v,v_0}_V
  \end{align*}
  とおく.
  このとき, (内積の定義\Cref{def:inner:item:line}より)
  $\beta_{v_0}$は線型写像である.
  つまり,
  \begin{align*}
    \beta_{v_0} \in \Hom_\RR(V,\RR) =V^\ast
  \end{align*}
  である.
  \begin{align*}
    \beta\colon
    V&\to V^\ast\\
    u&\mapsto \beta_{u}
  \end{align*}
  とおくと,
  (\Cref{def:inner:item:line:2:sum,def:inner:item:line:2:sca}より)
  $\beta$は線型写像である.
  (余談: $\beta$のことを内積を第2成分に関して
  カリー化した関数ということがある.)

  $V$は有限次元であるとする.
  $V$は有限次元であるとすると,
  $\beta$は同型写像である.
  つまり,
  \begin{align*}
    \forall \varphi \in V^\ast,
    \exists ! v_0 \in V \text{ s.t. }
    \beta_{v_0} = \varphi
  \end{align*}
  が成り立つ.
  これは, 言い換えると,
  \begin{align*}
    \forall \varphi \in V^\ast,
    \exists ! v_0 \in V \text{ s.t. }
    \forall v\in V, \varphi(v)=\Braket{v,v_0}_V
  \end{align*}
  となる.

\begin{remark}
  $V$が有限次元のとき,
  $\beta$は
  $V$と$V^\ast$の同型を与えるが,
  これは内積を取り替えると,
  異なる同型写像となる.
  内積を1つ決めるというのは,
  $V$と$V^\ast$の間の同型写像を1つ固定するということを意味する.
\end{remark}

$W$を有限次元内積空間とし,
$V$と同様に,
$w_0\in W$に対して,
\begin{align*}
  \beta_{w_0}\colon
  W&\to \RR\\
  w&\mapsto \Braket{w,w_0}_W
\end{align*}
とする.
$f\colon V\to W$を線型写像とする.
つまり, $f\in \Hom_\RR(V,W)$とする.
このとき
\begin{align*}
  \transposed{f}\colon
  W^\ast &\to V^\ast\\
  \varphi &\mapsto \varphi\circ f
\end{align*}
と定義され$\transposed{f}\in \Hom(W^\ast,V^\ast)$であった.

$w\in W$とする.
$\beta_w\in W^\ast$であるので,
$\transposed{f}(\beta_w)\in V^\ast$
である.
$V$が有限次元であるので,
$\beta\colon V\to V^\ast$は同型写像である.
したがって,
$\beta_{v_0}=\transposed{f}(\beta_w)$
を満たす$v_0\in V$が定まる.
この$v_0$を,
\begin{align*}
  f^\top(w) \in V
\end{align*}
とおく.
\begin{align*}
  f^\top\colon W&\to V\\
  w&\mapsto f^\top(w)
\end{align*}
という写像が定まる.
定義から,
\begin{align*}
  \transposed{f}(\beta_w)=\beta_{f^\top (w)}
\end{align*}
であるので,
$v\in V$, $w\in W$に対し,
\begin{align*}
  \Braket{f(v),w}_W
  &=\beta_w(f(v))\\
  &=(\beta_w \circ f)(v)\\
  &=(\transposed{f}(\beta_w))(v)\\
  &=\beta_{f^\top (w)}(v)\\
  &=\Braket{v,f^\top (w)}_V
\end{align*}
が成り立つ.
つまり,
\begin{align*}
  f^\top \colon W &\to V\\
  w&\mapsto f^\top (w)
\end{align*}
は
\begin{align*}
  v\in V, w\in W
  \implies
  \Braket{f(v),w}_W
  =\Braket{v,\beta_{f^\top (w)}}_V
\end{align*}
を満たす.
さらに,
$f^\top\in \Hom(W,V)$
であることもわかる.
$f^\top$も$f$の転置と呼ぶ.
\begin{remark}
  $f^\top \colon W \to V$
  は内積による$V$から$V^\ast$への同型$\beta$に依存している.
\end{remark}


\begin{remark}
  線型写像$f\colon V\to W$に対し,
  \begin{align*}
    \text{$f\colon V\to W$が等長写像}
    &\iff
    \forall v,v'\in V, \Braket{f(v),f(v')}_W=\Braket{v,v'}_V\\
    &\iff
    \forall v,v'\in V, \Braket{v,f^\top (f(v'))}_V=\Braket{v,v'}_V\\
    &\iff
    \forall v,v'\in V, \Braket{v,(f^\top \circ f)(v')}_V=\Braket{v,v'}_V\\
    &\iff
    \forall v'\in V, (f^\top \circ f)(v')=v'\\
    &\iff
    f^\top \circ f=\id_V
  \end{align*}
  である. $f\colon V\to V$について考えると,
  $f$が等長写像であることと$f^\top=f^{-1}$であることが
  同値であることがわかる.
\end{remark}


\begin{remark}
  $V$, $W$を有限次元ベクトル空間とすると,
  線型写像$f\colon V\to W$に対し,
  $f$の転置
  $\transposed{f}\in\Hom_\RR(W^\ast,V^\ast)$
  が定まる.
  この転置$\transposed{f}$は内積などによらずに定まる.
  しかし, $\transposed{f}$は$W$から$V$への写像ではない.
  一方$V$, $W$に内積があれば,
  $f\colon V\to W$の転置,
  $f^\top\in\Hom_\RR(W,V)$
  が定まる.
  この転置$f^\top$は$W$から$V$への写像ではあるものの,
  $f$に対して得られる転置$f^\top$は
  内積が変われば違うものとなる.

  すでに見たように,
  $V$, $W$に内積があれば,
  内積を通じて,
  $V$と$V^\ast$の同型$\beta$や,
  $W$と$W^\ast$の同型$\beta$が得られる.
  $f\in \Hom_\RR(V,W)$を一つ固定する.
  このとき$f^\top \in \Hom_\RR(W,V)$であるので,
  $w\in W$に対し, $f^\top(w)\in V$である.
  したがって, $\beta_{f^\top(w)}\in V^\ast$である.
  一方, $w\in W$であるので, $\beta_w\in W^\ast$である.
  $\transposed{f}\in \Hom_\RR(W^\ast,V^\ast)$であるので,
  $\transposed{f}(\beta_w)\in V^\ast$である.
  $V^\ast$の2つの元$\beta_{f^\top(w)}$と$\transposed{f}(\beta_w)$が得られたが,
  $v\in V$に対し,
  \begin{align*}
    \beta_{f^\top(w)}(v)
    &=\Braket{v,f^\top(w)}_V\\
    &=\Braket{f(v),w}_W\\
    (\transposed{f}(\beta_w))(v)
    &=(\beta_w\circ f)(v)\\
    &=\beta_w(f(v))\\
    &=\Braket{f(v),w}_W
  \end{align*}
  となるので,
  $\beta_{f^\top(w)}=\transposed{f}(\beta_w)$となる.
  つまり図式
  \begin{align*}
    \begin{CD}
      W @>f^\top>> V\\
      @V\beta VV @VV\beta V\\
      W^\ast @>>\transposed{f}> V^\ast
    \end{CD}
  \end{align*}
  が可換となっている.
\end{remark}


\section{数ベクトル空間と行列}
数ベクトル空間と行列についてまとめている.
$m$行$n$列の行列は,
通常なら数を表の様に並べたデータとして定義されるが,
ここでは, 与えられた数により定まる
$\RR^n$から$\RR^m$への線型写像として定義している.


\subsection{数ベクトルと行列}
\begin{align*}
  \RR^n=
  \Set{\begin{pmatrix}x_1\\\vdots\\x_n\end{pmatrix}|x_i\in \RR}
\end{align*}
は実ベクトル空間であった.
これを$n$次元(実)数ベクトル空間と呼ぶ.
$\RR^n$の元
\begin{align*}
  \xx=\begin{pmatrix}x_1\\\vdots\\x_n\end{pmatrix}
\end{align*}
を$n$項(実)数ベクトルと呼ぶ.
\begin{remark}
  次元は空間に対して, 項は元に対して用いている.
\end{remark}

すべての成分が0である$n$項数ベクトルを$\zzero$で表す.
\begin{definition}
  $i$行目のみ1で他は0である$n$項数ベクトル
  \begin{align*}
    \ee_i=\begin{pmatrix}0\\\vdots\\0\\1\\0\\\vdots\\0\end{pmatrix}
  \end{align*}
  を$\RR^n$の第$i$基本ベクトルといい,
  $(\ee_1,\ldots,\ee_n)$を$\RR^n$の標準基底と呼ぶ.
\end{definition}

$V$を実ベクトル空間とし,
$v_1,\ldots,v_n\in V$とする.
このとき,
\begin{align*}
  \pi_{v_1,\ldots,v_n} \colon \RR^n &\to V\\
  \begin{pmatrix}x_1\\\vdots\\x_n\end{pmatrix}&\mapsto
    x_1v_1+\cdots+x_nv_n
\end{align*}
とおく.
$\pi_{v_1,\ldots,v_n}\in \Hom_\RR(\RR^n,V)$である.
定義から,
$\pi_{v_1,\ldots,v_n}(\ee_i)=v_i$である.

\begin{definition}
  $V$を実ベクトル空間とし,
  $v_1,\ldots,v_n\in V$とする.
  $\pi_{v_1,\ldots,v_n}$が単射であるとき,
  $(v_1,\ldots,v_n)$は一次独立であるという.
\end{definition}
\begin{remark}
  $V$を実ベクトル空間とし,
  $v_1,\ldots,v_n\in V$とする.
  このとき, 以下は同値:
  \begin{enumerate}
  \item $\pi_{v_1,\ldots,v_n}$が単射.
  \item $\Ker(\pi_{v_1,\ldots,v_n})=\Set{\zzero}$.
  \item $x_1v_1+\cdots+x_nv_n=0_V \implies x_1=\cdots=x_n=0$.
  \end{enumerate}
\end{remark}

\begin{definition}
  $V$を実ベクトル空間とし,
  $v_1,\ldots,v_n\in V$とする.
  $\pi_{v_1,\ldots,v_n}$が全射であるとき,
  $(v_1,\ldots,v_n)$は$V$の生成系であるという.
\end{definition}
\begin{remark}
  $V$を実ベクトル空間とし,
  $v_1,\ldots,v_n\in V$とする.
  このとき, 以下は同値:
  \begin{enumerate}
  \item $\pi_{v_1,\ldots,v_n}$が全射.
  \item $\Img(\pi_{v_1,\ldots,v_n})=V$.
  \item $V=\Set{x_1v_1+\cdots+x_nv_n|x_i\in \RR}$.
  \end{enumerate}
\end{remark}

\begin{definition}
  $V$を実ベクトル空間とし,
  $v_1,\ldots,v_n\in V$とする.
  $(v_1,\ldots,v_n)$が一次独立であり$V$の生成系であるとき,
  $(v_1,\ldots,v_n)$を$V$の基底と呼ぶ.
\end{definition}
\begin{remark}
  $V$を実ベクトル空間とし,
  $v_1,\ldots,v_n\in V$とする.
  このとき, 以下は同値:
  \begin{enumerate}
  \item $(v_1,\ldots,v_n)$が$V$の一次独立な生成系.
  \item $\pi_{v_1,\ldots,v_n}$が全単射.
  \item $\pi_{v_1,\ldots,v_n}$が同型写像.
  \end{enumerate}
\end{remark}

\begin{remark}
  $(v_1,\ldots,v_n)$が$V$の基底であるとき,
  $\pi_{v_1,\ldots,v_n}$は$\RR^n$から$V$への同型写像である.
  $\RR^n$と同型なベクトル空間を$n$次元実ベクトル空間と呼ぶ.
\end{remark}

\begin{remark}
  $n$次元実ベクトル空間は,
  $\RR^n$と同型である.
  したがって,
  有限次元ベクトル空間としては$\RR^n$を考えれば十分である.
\end{remark}
\begin{remark}
  $\varphi \in \Hom_\RR(\RR^n,V)$に対し,
  次は同値である:
  \begin{enumerate}
  \item $\varphi=\pi_{v_1,\ldots,v_n}$
  \item $\varphi(\ee_1)=v_1,\ldots,\varphi(\ee_n)=v_n$
  \end{enumerate}
  言い換えると,
  \begin{align*}
    \pi\colon V\times\cdots \times V &\to \Hom_\RR(\RR^n,V)\\
    (v_1,\ldots,v_n)&\mapsto
    \pi_{v_1,\ldots,v_n}
  \end{align*}
  が全単射である.
\end{remark}

\begin{definition}
$(i,j)\in\Set{1,\ldots,m}\times \Set{1,\ldots,n}$に対し,
$a_{i,j}\in\RR$を固定する.
このとき,
\begin{align*}
  \RR^n&\to \RR^m\\
  \begin{pmatrix}x_1\\\vdots\\x_n\end{pmatrix}&\mapsto
    \begin{pmatrix}\sum_{k=1}^n a_{1,k}x_k\\\vdots\\\sum_{k=1}^n a_{m,k}x_k\end{pmatrix}
\end{align*}
という写像は線形写像である.
この写像を,
\begin{align*}
  \begin{pmatrix}
    a_{1,1}&\cdots &a_{1,n}\\
    \vdots& &\vdots\\
    a_{m,1}&\cdots &a_{m,n}
  \end{pmatrix}
\end{align*}
とか,
\begin{align*}
  (a_{i,j})_{i,j}
\end{align*}
など%
\footnote{$(a_{i,j})_{i,j}$という標記は, $(\quad)_{i,j}$とカッコの外に$i,j$と書くことで, $i$が`行'に関わる添字, $j$が`列'に関わる添字であることを表している.
$(\quad)_{j,i}$と書いたら$j$が行に関わる添字, $i$が列に関わる添字であることを表す.
したがって, $(a_{i,j})_{i,j}=(a_{j,i})_{j,i}=(a_{k,l})_{k,l}$である.}%
と表し,
サイズが$(m,n)$の実行列と呼ぶ.
写像を定める係数$a_{i,j}$を
$(a_{i,j})_{i,j}$の$(i,j)$-成分と呼ぶ.
行列の和, スカラー倍, 積を,
線型写像の和, スカラー倍, 合成として定義する.
線型写像$(a_{i,j})_{i,j}$が同型写像であるとき,
$(a_{i,j})_{i,j}$は正則であるという.
\end{definition}
\begin{remark}
  サイズが$(m,n)$の実行列のことを,
以下では$(m,n)$-行列と呼ぶ.
\end{remark}
\begin{remark}
\begin{align*}
\id_{\RR^n} \colon \RR^n&\to \RR^n\\
 \xx &\mapsto \xx
\end{align*}
  に対応する行列は,
\begin{align*}
  (\delta_{i,j})_{i,j}
\end{align*}
である.
ただし
\begin{align*}
  \delta_{i,j}=
  \begin{cases}
    1&i=j\\
    0&i\neq j.
  \end{cases}
\end{align*}
これは単位行列と呼ばれる.
ここでは$E_n$で表す.
\end{remark}


\begin{remark}
\begin{align*}
  \underline{\zzero} \colon \RR^n&\to \RR^m\\
  \xx &\mapsto \zzero
\end{align*}
  に対応する行列は,
\begin{align*}
  (0)_{i,j}
\end{align*}
である.
これは零行列と呼ばれる.
ここでは$O_{m,n}$で表す.
\end{remark}
\begin{remark}
  行列$A$が``実''行列かどうかは,
  $A$を$\RR^n$から$\RR^m$への写像として考えているか否かで
  (本来は) 判断するべきもの.
  各成分が実数であっても,
  $\CC^n\to \CC^m$と思うのであれば
  複素行列である.
\end{remark}



\begin{prop}
  $A=(a_{i,j})_{i,j}$を$(m,n)$-行列,
  $B=(b_{i,j})_{i,j}$を$(m',n')$-行列とする.
  \begin{enumerate}
  \item $A$と$B$の和は, $(m,n)=(m',n')$であるときに定義でき,
    \begin{align*}
      A+B=(a_{i,j}+b_{i,j})_{i,j}.
    \end{align*}
  \item $c\in \RR$に対し,
    \begin{align*}
      cA=(ca_{i,j})_{i,j}.
    \end{align*}
  \item $A$と$B$の積は, $n=m'$であるときに定義でき,
    \begin{align*}
      AB=(\sum_{k=1}^n a_{i,k}b_{k,j})_{i,j}.
    \end{align*}
  \end{enumerate}
\end{prop}
\begin{remark}
  $A=(a_{i,j})_{i,j}$をサイズが$(m,n)$である行列とする.
  \begin{align*}
    \aaa^{(j)}=
    \begin{pmatrix}
      a_{1,j}\\\vdots\\a_{m,j}
    \end{pmatrix}
  \end{align*}
  という
  $m$項数ベクトルを考える.
\begin{align*}
  A\colon \RR^n&\to \RR^m\\
  \begin{pmatrix}x_1\\\vdots\\x_n\end{pmatrix}&\mapsto
    \begin{pmatrix}\sum_{k=1}^n a_{1,k}x_k\\\vdots\\\sum_{k=1}^n a_{m,k}x_k\end{pmatrix}
      =
      x_1\begin{pmatrix}a_{1,1}\\\vdots\\a_{m,1}\end{pmatrix}
      +\cdots+
      x_1\begin{pmatrix}a_{1,n}\\\vdots\\a_{m,n}\end{pmatrix}
\end{align*}
であるので,
\begin{align*}
  A=\pi_{\aaa^{(1)},\ldots,\aaa^{(n)}}
\end{align*}
である.
したがって
\begin{align*}
  \Hom_\RR(\RR^n,\RR^m)
  &=\Set{\pi_{\aaa^{(1)},\ldots,\aaa^{(n)}}|\aaa^{(j)}\in\RR^m}\\
  &=\Set{A|\text{$A$は$(m,n)$-行列}}
\end{align*}
であるので,
$\RR^n$から$\RR^m$への線型写像は
(サイズが$(m,n)$の)行列だけ考えれば十分であることがわかる.
\end{remark}

\begin{align*}
  (\RR^n)^\ast
  &=\Hom_\RR(\RR^n,\RR)\\
  &=\Set{A|\text{$(1,n)$-行列}}\\
  &=\Set{\begin{pmatrix}a_1&\cdots&a_n\end{pmatrix}|a_i\in\RR}
\end{align*}
である.
\begin{align*}
  \ee_1^\top &= \begin{pmatrix}1&0&0&\cdots&0&0\end{pmatrix}\\
  \ee_2^\top &= \begin{pmatrix}0&1&0&\cdots&0&0\end{pmatrix}\\
      &\vdots\\
  \ee_n^\top &= \begin{pmatrix}0&0&0&\cdots&0&1\end{pmatrix}
\end{align*}
とおくと,
$(\ee_1^\top,\ldots,\ee_n^\top)$
は$(\RR^n)^\ast$の基底である.
この基底を$(\ee_1,\ldots,\ee_n)$の双対基底と呼ぶ.
\begin{align*}
  \pi_{\ee_1^\top,\ldots,\ee_n^\top}\colon
  \RR^n&\to (\RR^n)^\ast\\
  \begin{pmatrix}x_1\\\vdots\\x_n\end{pmatrix}&\mapsto
    x_1\ee_1^\top+\cdots+x_n\ee_n^\top=
    \begin{pmatrix}x_1&\cdots&x_n\end{pmatrix}
\end{align*}
は同型写像である.
この同型写像を使って,
$\RR^n$と$(\RR^n)^\ast$を同一視できる.

\subsection{内積と直交行列}
\begin{align*}
  \Braket{\bullet,\bullet}_{\RR^n}\colon \RR^n \times \RR^n &\to \RR\\
  (\begin{pmatrix}x_1\\\vdots\\x_n\end{pmatrix},\begin{pmatrix}y_1\\\vdots\\y_n\end{pmatrix})
      &\mapsto x_1y_1+\cdots+x_ny_n
\end{align*}
とおき,
$\RR^n$の標準内積とかユークリッド内積と呼ぶ.
\begin{prop}
  $(\RR^n,\Braket{,}_{\RR^n})$は内積空間である.
  つまり$\Braket{,}$は$\RR^n$上の
  非退化な正定値対称双線型形式である.
\end{prop}
\begin{definition}
  $(m,n)$-行列$A=(a_{i,j})_{i,j}$の
  転置$A^\top$を,
  $A\in\Hom_\RR(\RR^n,\RR^m)$としての転置$A^\top\in\Hom_\RR(\RR^m,\RR^n)$
  として定義する.
\end{definition}
\begin{prop}
  $(m,n)$-行列$A=(a_{i,j})_{i,j}$に対し,
  $A^\top\in \Hom_\RR(\RR^m,\RR^n)$であり,
  $A^\top=(a_{j,i})_{i,j}$である.
\end{prop}
\begin{proof}
  $A'=(a_{j,i})_{i,j}$とおく.
  $A^\top$は$v\in\RR^n$, $w\in \RR^m$に対し,
  \begin{align*}
    \Braket{w,A(v)}_{\RR^m} =
        \Braket{A^\top(w),v}_{\RR^n} 
  \end{align*}
  を満たすものであったので, $A'$がこの等式を満たすことを示せばよい.
  \begin{align*}
    w&=\begin{pmatrix}y_1\\\vdots\\y_m\end{pmatrix},&
    v&=\begin{pmatrix}x_1\\\vdots\\x_n\end{pmatrix}
  \end{align*}
  とすると,
  \begin{align*}
    \Braket{w,A(v)}_{\RR^m}
    &=\Braket{\begin{pmatrix}y_1\\\vdots\\y_m\end{pmatrix},A(\begin{pmatrix}x_1\\\vdots\\x_n\end{pmatrix})}_{\RR^m}\\
    &=\Braket{\begin{pmatrix}y_1\\\vdots\\y_m\end{pmatrix},\begin{pmatrix}\sum_{k=1}^na_{1,k}x_k\\\vdots\\\sum_{k=1}^na_{m,k}x_k\end{pmatrix}}_{\RR^m}\\
    &=\sum_{l=1}^m \sum_{k=1}^n y_{l}a_{l,k} x_k,\\
    \Braket{A'(w),v)}_{\RR^n}
    &=\Braket{A(\begin{pmatrix}y_1\\\vdots\\y_m\end{pmatrix}),\begin{pmatrix}x_1\\\vdots\\x_n\end{pmatrix}}_{\RR^n}\\
    &=\Braket{\begin{pmatrix}\sum_{l=1}^{m}a_{l,1}y_l\\\vdots\\\sum_{l=1}^{m}a_{l,n}y_l\end{pmatrix},\begin{pmatrix}x_1\\\vdots\\x_n\end{pmatrix}}_{\RR^n}\\
    &=\sum_{k=1}^n\sum_{l=1}^{m}x_ka_{l,k}y_l\\
    &=\sum_{l=1}^{m}\sum_{k=1}^ny_la_{l,k}x_k.
  \end{align*}
\end{proof}

\begin{remark}
  $\transposed{A}$から$A^\top$を定義する際,
  内積を通じた$V$から$V^\ast$への同型$\beta$が使われるが,
  この場合は,
  \begin{align*}
    \beta\colon \RR^n &\to (\RR^n)^\ast \\
    \begin{pmatrix}x_1\\\vdots\\x_n\end{pmatrix}
      &\mapsto
      \begin{pmatrix}x_1&\cdots&x_n\end{pmatrix}
  \end{align*}
  である.
\end{remark}

\begin{definition}
  $(V,\Braket{,}_V)$を内積空間とし,
  $v_1,\ldots,v_n\in V$とする.
  次の条件を満たすとき,
  $(v_1,\ldots,v_n)$は正規直交系(orthonormal system)であるという:
  \begin{enumerate}
    \item $v_i,v_j\in V\implies \Braket{v_i,v_j}_V=\delta_{i,j}$.
  \end{enumerate}
\end{definition}
\begin{remark}
  $(v_1,\ldots,v_n)$は正規直交系であるなら,
  $(v_1,\ldots,v_n)$は一次独立である.
\end{remark}
\begin{definition}
  $(V,\Braket{,}_V)$を内積空間とし,
  $v_1,\ldots,v_n\in V$とする.
  次の条件を満たすとき,
  $(v_1,\ldots,v_n)$は$V$の完備正規直交系%
  \footnote{完全正規直交系と呼ばれることが多いように思うが, 完全という語はperfectなどの訳語としても用いられることもあり, completeの訳語であることを意識して完備という語を, ここでは用いる.}%
  (complete orthonormal system)であるという:
  \begin{enumerate}
    \item $(v_1,\ldots,v_n)$は正規直交系
    \item $v\in V$に対し次は同値:
      \begin{enumerate}
      \item $\forall i,\  \Braket{v,v_j}_V=0$.
      \item $v=0_V$.
      \end{enumerate}
  \end{enumerate}
\end{definition}
\begin{remark}
  \begin{align*}
    v=0_V\implies \forall i,\  \Braket{v,v_j}_V=0
  \end{align*}
  は一般に成り立つので,
  完備正規直交系であることの定義としては,
  逆のみが書かれることもある.
\end{remark}
\begin{example}
  $\RR^n$について考える.
  \begin{align*}
    \Braket{\ee_i,\ee_j}_{\RR^n}=\delta_{i,j}
  \end{align*}
  である.
  また,
  \begin{align*}
    \braket{\begin{pmatrix}x_1\\\vdots\\x_n\end{pmatrix},\ee_i}_{\RR^n}
    =x_i
  \end{align*}
  であるので, $\ee_i$との標準内積が0であることは,
  第$i$成分が0であることを意味する.
  したがって,
  すべての$i$に対し$\Braket{v,\ee_i}_{\RR^n}$を満たす$v\in\RR^n$は,
  $\zzero$のみであることがわかる.
  したがって,
  $(\ee_1,\ldots,\ee_n)$は
  $(\RR^n,\Braket{}_{\RR^n})$
  完備正規直交系である.
\end{example}
\begin{lemma}
  $(V,\Braket{,}_V)$を内積空間とし,
  $(v_1,\ldots,v_n)$は$V$の完備正規直交系であるとする.
  このとき,
  \begin{align*}
    \pi_{v_1,\ldots,v_n}\colon \RR^n\to V
  \end{align*}
  は等長な線型同型写像である.
  つまり,
  $(\RR^n,\Braket{}_{\RR^n})$と
  $(V,\Braket{,}_V)$は内積空間として同じものだと思える.
\end{lemma}


\begin{theorem}[Gram--Schmidtの直交化法]
  $(V,\Braket{,}_V)$を内積空間とし,
  $v_1,\ldots,v_n\in V$とする.
  $(v_1,\ldots,v_n)$が一次独立であることを仮定する.
  このとき, 以下の様に$\bar v_1,\ldots, \bar v_n$を定義する:
  \begin{align*}
    v'_1&=v_1&
    \bar v_1&=\frac{1}{\Braket{v'_1,v'_1}_V}v_1\\
    v'_2&=v_2-\Braket{v_2,\bar v_1}_V \bar v_1&
    \bar v_2&=\frac{1}{\Braket{v'_2,v'_2}_V}v_2\\
    v'_3&=v_3-\sum_{i=1}^{2}\Braket{v_3,\bar v_i}_V \bar v_i&
    \bar v_2&=\frac{1}{\Braket{v'_3,v'_3}_V}v_3\\
    &\vdots&&\vdots\\
    v'_k&=v_k-\sum_{i=1}^{k-1}\Braket{v_k,\bar v_i}_V \bar v_i&
    \bar v_k&=\frac{1}{\Braket{v'_k,v'_k}_V}v_k\\
    &\vdots&&\vdots.
  \end{align*}
  このとき,
  $(\bar v_1,\ldots, \bar v_n)$は正規直交系である.
\end{theorem}
\begin{remark}
  $(V,\Braket{,}_V)$を有限次元内積空間とし,
  $(v_1,\ldots,v_n)$を$V$の基底とする.
  $(\bar v_1,\ldots, \bar v_n)$を,
  Gram--Schmidtの直交化法で得られた正規直交系であるとする.
  このとき,
  $(\bar v_1,\ldots, \bar v_n)$は$V$の完全正規直交系である.
  よって,
  \begin{align*}
    \pi_{\bar v_1,\ldots,\bar v_n}\colon \RR^n\to V
  \end{align*}
  により,
  $(\RR^n,\Braket{}_{\RR^n})$と
  $(V,\Braket{,}_V)$は内積空間として同じものだと思える.  
\end{remark}
\begin{definition}
  \begin{align*}
    O(n)=O(\RR^n,\Braket{}_{\RR^n})
  \end{align*}
  とおき, $n$次直交群と呼ぶ.
  $O(n)$の元を直交行列と呼ぶ.
\end{definition}

\begin{remark}
    \begin{align*}
    O(n)=\Set{A|
      \begin{array}{c}
        \text{$A$は$(n,n)$-行列}\\
        A^\top A =E_n
      \end{array}
    }
  \end{align*}
    である.

    $A=(a_{i,j})_{i,j}\in O(n)$とする.
  \begin{align*}
    \aaa^{(j)}=
    \begin{pmatrix}
      a_{1,j}\\\vdots\\a_{n,j}
    \end{pmatrix}
  \end{align*}
  とおく.
  $A^\top A=E_n$であるが,
  右辺の$(i,j)$成分は,
  \begin{align*}
    \sum_{k=1}^n
    a_{k,i}a_{k,j}
    =\Braket{\aaa^{(i)},\aaa^{(j)}}_{\RR^n}
  \end{align*}
  である. $E_n$の$(i,j)$成分は$\delta_{i,j}$であるので,
  \begin{align*}
    \Braket{\aaa^{(i)},\aaa^{(j)}}_{\RR^n}=\delta_{i,j}
  \end{align*}
  である.
  つまり,
  $(\aaa^{(1)},\ldots,\aaa^{(n)})$は正規直交系である.

  逆に, $\RR^n$の正規直交系
  $(\aaa^{(1)},\ldots,\aaa^{(n)})$が与えられたとき,
  \begin{align*}
    \begin{pmatrix}
      a_{1,j}\\\vdots\\a_{n,j}
    \end{pmatrix}
    =\aaa^{(j)}
  \end{align*}
  とおけば,
  $A=(a_{i,j})_{i,j}$は直交行列である.
\end{remark}
