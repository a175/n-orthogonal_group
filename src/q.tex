% !TeX root =./x2.tex
% !TeX program = pdfpLaTeX

\chapter{線形代数の復習}

\section{抽象実ベクトル空間と線形写像}
和と実数によるスカラー倍が備わった空間を
実ベクトル空間と呼ぶ.
\begin{definition}
  $V$を集合とし,
  $+$, $\cdot$を二項演算
  \begin{align*}
    \bullet + \bullet \colon V\times V &\to V\\
    (v,w)&\mapsto v+w\\
    \bullet \cdot \bullet \colon \RR\times V &\to V\\
    (a,w)&\mapsto a\cdot w
  \end{align*}
  とし,
  $0_V$
  とする.
  以下の条件を満たすとき,
  $(V,+,\cdot,0_V)$を実ベクトル空間と呼ぶ:
  \begin{enumerate}
  \item $v,w,u\in V$に対し以下が成り立つ:
    \begin{enumerate}
    \item $(v+w)+u=v+(w+u)$.
    \item $v+w=w+v$.
    \item $v+0_V=v$.
    \item $v+(-1)\cdot v=0_V$.
    \end{enumerate}
  \item $v\in V$, $a,b\in\RR$に対し
    以下が成り立つ:
    \begin{enumerate}
    \item $(ab)\cdot v=a\cdot (b\cdot v)$.
    \item $1\cdot v=v$.
    \end{enumerate}      
  \item $v,w \in V$, $a,b\in\RR$に対し
    以下が成り立つ:
    \begin{enumerate}
    \item $(a+b)\cdot v=(a\cdot v) + (b\cdot v)$.
    \item $a\cdot (v+w)=(a\cdot v) + (a\cdot w)$.
    \end{enumerate}      
  \end{enumerate}
\end{definition}
\begin{remark}
  $(V,+,\cdot,0_V)$が実ベクトル空間であるとき,
  \begin{align*}
    0\cdot v
    &=(1-1)\cdot v\\
    &=1\cdot v + (-1)\cdot v\\
    &=v+ (-1)\cdot v= 0_V
  \end{align*}
  であるので,
  任意の$v\in V$に対し, $0v=0_V$である.
\end{remark}
\begin{remark}
  以下では, 文脈上誤解が生じないときには,
  $(V,+,\cdot,0_V)$が実ベクトル空間であることを単に
  $V$がベクトル空間であるといい,
  $0_V$のことを単に$0$と書いたりする.
  また, スカラー倍の際$\cdot$を省略し, $a\cdot v$を$av$の様に書くこともある.
\end{remark}
\begin{remark}
  $(-1)v$を$-v$, $w+(-v)$を$w-v$のように略記する.
\end{remark}


和とスカラー倍とコンパチブルな写像を線形写像と呼ぶ.
\begin{definition}
  $(V,+,\cdot,0_V)$, $(W,\oplus,\odot,0_W)$を実ベクトル空間とし,
  $f$を$V$から$W$への写像とする.
  以下の条件を満たすとき,
  $f$は$(V,+,\cdot,0_V)$から$(W,\oplus,\odot,0_W)$への($\RR$上の)線形写像であるという:
  \begin{enumerate}
  \item $v,v'\in V\implies f(v+v')=f(v)\oplus f(v')$.
  \item $a\in \RR$, $v\in V \implies f(a\cdot v)=a\odot f(v)$.
  \end{enumerate}
\end{definition}
\begin{remark}
  以下では, 文脈上誤解が生じないときには,
  $f$は$(V,+,\cdot,0_V)$から$(W,\oplus,\odot,0_W)$への($\RR$上の)線形写像であることを,
  単に$V$から$W$への線形写像であるという.
  また,
  本来であれば,
  線形写像$f\colon (V,+,\cdot,0_V)\to(W,\oplus,\odot,0_W)$
  などと書き表すべきであるが,
  線形写像$f\colon V\to W$などと書く.
\end{remark}
\begin{remark}
  $V$を実ベクトル空間とする.
  $V$上の恒等写像$\id_V$は線形写像である.
\end{remark}
\begin{definition}
  実ベクトル空間$V$, $W$に対し,
  \begin{align*}
    \Hom_\RR(V,W)
    =\Set{f\colon V\to W \text{: 線形}}
  \end{align*}
  とおく.
\end{definition}
\begin{remark}
  線形写像の合成は線形である.
  つまり,
  \begin{align*}
  f\in \Hom_\RR(V,W),\ 
  g\in \Hom_\RR(W,U)
  \implies
  g\circ f \in \Hom_\RR(W,U).
  \end{align*}
\end{remark}
\begin{definition}
  $V$, $W$をベクトル空間とし,
  $f\colon V\to W$を線形写像とする.
  次の条件を満たすとき,
  $f$は$V$から$W$への(線形)同型写像であるという:
  \begin{itemize}
  \item 次の条件を満たす線形写像$g\colon W\to V$が存在する:
    \begin{align*}
      g\circ f &=\id_V,\\
      f\circ g &=\id_W.
    \end{align*}
  \end{itemize}
  また, $V$から$W$への同型写像が存在するとき,
  $V$と$W$は(線形)同型であるといい$V\simeq W$と書く.
  つまり$V\simeq$と次は同値である:
  \begin{itemize}
  \item 次の条件を満たす線形写像$f\colon V\to W$と$g\colon W\to V$が存在する:
    \begin{align*}
      g\circ f &=\id_V,\\
      f\circ g &=\id_W.
    \end{align*}
  \end{itemize}
\end{definition}
\begin{remark}
  $V$と$W$が同型であるとき,
  $V$で成り立つことは同型写像$f$を通して$W$に翻訳できるので,
  $V$と$W$はベクトル空間として同じものであると思える.
\end{remark}
\begin{remark}
  線形写像$f\colon V\to W$が全単射であるとき,
  $f$の逆写像$f^{-1}$も線形である.
  したがって,
  $f\colon V\to W$が同型写像であることと,
  $f\colon V\to W$が全単射な線形写像であることは同値である.
\end{remark}
\begin{example}
  数ベクトルのなす集合
  \begin{align*}
    \RR^n=\Set{\begin{pmatrix}x_1\\\vdots\\x_n\end{pmatrix}|x_i\in\RR}
  \end{align*}
  は実ベクトル空間である. 
\end{example}
\begin{example}
  複素数のなす集合
  \begin{align*}
    \CC=\Set{x+y\sqrt{-1}|x,y\in\RR}
  \end{align*}
  は実ベクトル空間である.
  これは, 複素(数)平面と呼ばれる.
\end{example}
\begin{example}
  実数のなす集合$\RR=\RR^1$
  は実ベクトル空間である.
\end{example}

\begin{definition}
  $(V,+,\cdot,0_V)$,
  $(W,+,\cdot,0_W)$をベクトル空間とする.
  $f,g\in\Hom_\RR(V,W)$に対し,
  写像$f+g$を次で定義する:
  \begin{align*}
    f+g\colon V&\to W\\
    v&\mapsto f(v)+g(v).
  \end{align*}
  $a\in \RR$,
  $f\in\Hom_\RR(V,W)$に対し,
  写像$a\cdot f$を次で定義する:
  \begin{align*}
    a\cdot f\colon V&\to W\\
    v&\mapsto a\cdot f(v).
  \end{align*}
  また, 写像$\underline{0_W}$を次で定義する:
  \begin{align*}
    \underline{0_W}\colon V&\to W\\
    v&\mapsto 0_W.
  \end{align*}
\end{definition}
\begin{prop}
  ベクトル空間
  $(V,+,\cdot,0_V)$,
  $(W,+,\cdot,0_W)$に対し,
  $\underline{0_W}\in \Hom_\RR(V,W)$である.
  $f,g\in\Hom_\RR(V,W)$, $a\in \RR$に対し,
  $f+g, a\cdot f\in \Hom_\RR(V,W)$である.
  また, 
  $(\Hom_\RR(V,W),+,\cdot,\underline{0_W})$
  は実ベクトル空間である.
\end{prop}
\begin{remark}
  \label{A:rem:linearmap:sum:sc}
  $\Hom_\RR(V,W)$がベクトル空間であるので,
  和とスカラー倍は分配則などを満たす.
  さらに, 合成とは以下を満たす:
  $f,f'\in\Hom_\RR(V,W)$,
  $g,g'\in\Hom_\RR(W,U)$,
  $a\in\RR$に対し,
  \begin{enumerate}
  \item $(g+g')\circ f=g\circ f+g'\circ f$,
  \item $g\circ (f+f')=g\circ f+g\circ f'$,
  \item $c(g\circ f)=(cg)\circ f=g\circ(cf)$.    
  \end{enumerate}
\end{remark}

\begin{definition}
  $V$をベクトル空間とする.
  \begin{align*}
    V^\ast &= \Hom_\RR(V,\RR)\\
    &=\Set{\varphi\colon V\to \RR\text{: 線形写像}}
  \end{align*}
  とおき, ベクトル空間$V^\ast$を$V$の双対空間と呼ぶ.
\end{definition}


$V$, $W$をベクトル空間とし,
$f$を$V$から$W$への線形写像とする.
$\varphi\in W^\ast$とすると,
$\varphi$は$W$から$\RR$への線形写像であるので,
\begin{align*}
  \varphi\circ f \colon V&\to \RR\\
  v&\mapsto \varphi(f(v))
\end{align*}
は線形である.
つまり, $\varphi\circ f \in \Hom_\RR(V,\RR)=V^\ast$である.
よって,
写像
\begin{align*}
  \Phi_f\colon W^\ast &\to V^\ast\\
  \varphi &\mapsto \varphi\circ f
\end{align*}
が定義できる.
\begin{lemma}
$V$, $W$をベクトル空間とし,
$f$を$V$から$W$への線形写像とし,
\begin{align*}
  \Phi_f\colon W^\ast &\to V^\ast\\
  \varphi &\mapsto \varphi\circ f
\end{align*}
とすると, $\Phi_f\in \Hom_\RR(W^\ast,V^\ast)$.
\end{lemma}
\begin{proof}
  $\varphi,\psi\in W^\ast$に対し,
  \begin{align*}
    \Phi_f(\varphi+\psi)&=(\varphi+\psi)\circ f \in W^\ast\\
    \Phi_f(\varphi)+\Phi_f(\psi)&=\varphi\circ f+\psi\circ f \in W^\ast
  \end{align*}
  である.
  $v\in V$に対し,
  \begin{align*}
    (\Phi_f(\varphi+\psi))(v)
    &=((\varphi+\psi)\circ f )(v)\\
    &=(\varphi+\psi)(f (v))\\
    &=\varphi(f (v))+\psi(f (v))\\
    (\Phi_f(\varphi)+\Phi_f(\psi))(v)
    &=(\varphi\circ f+\psi\circ f )(v)\\
    &=(\varphi\circ f)(v)+(\psi\circ f )(v)\\
    &=\varphi(f(v))+\psi( f (v))
  \end{align*}
  となるので%
  \footnote{\Cref{A:rem:linearmap:sum:sc}において証明せずに紹介した事実の証明をここで与えている.}%
  ,
  $\Phi_f(\varphi+\psi)=\Phi_f(\varphi)+\Phi_f(\psi)$.

  $a\in \RR$,
  $\varphi\in W^\ast$に対し,
  \begin{align*}
    \Phi_f(a\varphi)&=(a\varphi)\circ f \in W^\ast\\
    a\Phi_f(\varphi)&=a(\varphi\circ f)\in W^\ast
  \end{align*}
  である.
  $v\in V$に対し,
  \begin{align*}
    (\Phi_f(a\varphi))(v)
    &=((a\varphi)\circ f)(v)\\ 
    &=(a\varphi)(f(v))\\ 
    &=a(\varphi(f(v))\\ 
    (a\Phi_f(\varphi))(v)
    &=(a(\varphi\circ f))(v)\\
    &=a((\varphi\circ f)(v))\\
    &=a(\varphi(f(v)))
  \end{align*}
  となるので%
  \footnote{\Cref{A:rem:linearmap:sum:sc}において証明せずに紹介した事実の証明をここで与えている.}%
  ,
  $\Phi_f(a\varphi)=a\Phi_f(\varphi)$.
\end{proof}
\begin{definition}
  $V,W$をベクトル空間とする.
  $f\in \Hom_\RR(V,W)$に対し,
  $\transposed{f}\in \Hom_\RR(W^\ast,V^\ast)$を,
  \begin{align*}
    \transposed{f}\colon W^\ast &\to V^\ast\\
    \varphi &\mapsto \varphi\circ f
  \end{align*}
  で定義し, $f$の転置と呼ぶ.
\end{definition}

(余談)
\begin{align*}
  \Phi\colon \Hom_\RR(V,W)&\to \Hom_\RR(W^\ast,V^\ast)  \\
  f&\mapsto \Phi_f=\transposed{f}
\end{align*}
は線形である.
つまり,
\begin{align*}
  \Phi\in\Hom_\RR(\Hom_\RR(V,W),\Hom_\RR(W^\ast,V^ast)).
\end{align*}
\begin{proof}
  $f,f'\in \Hom_\RR(V,W)$とする.
  $\Phi_{f+f'},\Phi_{f}+\Phi_{f'}\in\Hom_\RR(W^\ast,V^\ast)$
  である.
  $\varphi\in W^\ast$に対し,
  \begin{align*}
    \Phi_{f+f'}(\varphi)
    &=(\transposed{(f+f')})(\varphi)\\
    &=\varphi\circ(f+f')\in V^\ast,\\
    (\Phi_{f}+\Phi_{f'})(\varphi)
    &=(\transposed{f}+\transposed{f'})(\varphi)\\
    &=(\transposed{f})(\varphi)+(\transposed{f'})(\varphi)\\
    &=\varphi\circ f+\varphi\circ f'\in V^\ast.
  \end{align*}
  $v\in V$に対し,
  \begin{align*}
    (\Phi_{f+f'}(\varphi))(v)
    &=(\varphi\circ(f+f'))(v)\\
    &=\varphi((f+f')(v))\\
    &=\varphi(f(v)+f'(v))\\
    &=\varphi(f(v))+\varphi(f'(v)),\\
    ((\Phi_{f}+\Phi_{f'})(\varphi))(v)
    &=(\varphi\circ f+\varphi\circ f')(v)\\
    &=(\varphi\circ f)(v)+(\varphi\circ f')(v)\\
    &=\varphi(f(v))+\varphi(f'(v)).
  \end{align*}
  よって, $\Phi_{f+f'}=\Phi_{f}+\Phi_{f'}$.
 
  $a\in\RR$,
  $f\in \Hom_\RR(V,W)$とする.
  $\Phi_{af},\Phi_{f}+\Phi_{f'}\in\Hom_\RR(W^\ast,V^\ast)$
  である.
  $\varphi\in W^\ast$に対し,
  \begin{align*}
    \Phi_{af}(\varphi)&=(\transposed{(a f)})(\varphi)=\varphi\circ(af)\in V^\ast\\
    (a\Phi_{f})(\varphi)&=a(\transposed{f}(\varphi))=a(\varphi\circ f)\in V^\ast.
  \end{align*}
  $v\in V$に対し,
  \begin{align*}
    (\Phi_{af}(\varphi))(v)
    &=(\varphi\circ(af)) (v)\\
    &=\varphi(af(v))\\
    &=a\varphi(f(v))\\
    ((a\Phi_{f})(\varphi))(v)
    &=(a(\varphi\circ f))(v)\\
    &=a(\varphi\circ f)(v))\\
    &=a(\varphi(f(v))).
  \end{align*}
  よって, $\Phi_{af}=a\Phi_{f}$.
\end{proof}


\section{実内積と等長写像}
\begin{definition}
  $V$を実ベクトル空間とする.
  \begin{align*}
    \Braket{\bullet,\bullet}_V
    \colon
    V\times V \to \RR\\
    (v,w) \mapsto\Braket{v,w}_V
  \end{align*}
  が以下の条件を満たすとき,
  $(V,\Braket{\bullet,\bullet}_V)$は(実)内積空間であるという:
  \begin{enumerate}
  \item
    \label{def:inner:item:line}
    $a\in\RR$, $v,w,u\in V\implies$
    \begin{align}
      \Braket{v+w,u}_V&=\Braket{v,u}_V+\Braket{w,u}_V,
      \label{def:inner:item:line:1:sum}
      \\
      \Braket{av,u}_V&=a\Braket{v,u}_V.
      \label{def:inner:item:line:1:sca}
    \end{align}
  \item
    \label{def:inner:item:sym}
    $v,w\in V \implies \Braket{v,w}_V=\Braket{w,v}_V$.
  \item 
    \label{def:inner:item:nondege}
    $v\in V$に対し次が成り立つ:
    \begin{align}
      v=0_V
      \iff
      \forall w\in V,\Braket{v,w}_V=0
      \label{def:inner:item:nondege:1}
    \end{align}
  \item
    \label{def:inner:item:semi}
    $v\in V \implies \Braket{v,v}_V\geq 0$.
  \item
    \label{def:inner:item:posi}
    $\Braket{v,v}_V=0\iff v=0_V$.
  \end{enumerate}
\end{definition}
\begin{remark}
  文脈上誤解がない場合には,
  $(V,\Braket{\bullet,\bullet}_V)$が内積空間であることを,
  単に$V$が内積空間であると言うこともある.
  $\Braket{\bullet,\bullet}_V$を$V$の内積と呼ぶ.
\end{remark}
\begin{remark}
  \Cref{def:inner:item:sym}を満たすことを
  $\Braket{\bullet,\bullet}_V$が対称であるという.
\end{remark}
\begin{remark}
  \Cref{def:inner:item:line}
  を満たすことを
  $\Braket{\bullet,\bullet}_V$が
  第一成分に関して線形
  であるという.
  \Cref{def:inner:item:sym,def:inner:item:line}から
  \begin{align}
    \Braket{u,v+w}_V&=\Braket{u,v}_V+\Braket{u,w}_V,
    \label{def:inner:item:line:2:sum}
    \\
    \Braket{u,av}_V&=a\Braket{u,v}_V    
    \label{def:inner:item:line:2:sca}
  \end{align}
  がわかる.
  \Cref{def:inner:item:line:2:sum,def:inner:item:line:2:sca}
  を満たすことを,
  $\Braket{\bullet,\bullet}_V$が
  第二成分に関して線形
  であるという.
  \Cref{def:inner:item:line:2:sum,def:inner:item:line:2:sca,def:inner:item:line:1:sum,def:inner:item:line:1:sca}
  を満たすことを,
  $\Braket{\bullet,\bullet}_V$が
  双線形
  であるという.
\end{remark}
\begin{remark}
  \Cref{def:inner:item:sym,def:inner:item:nondege:1}
  から,
  \begin{align}
      v=0_V
      \iff
      \forall w\in V,\Braket{w,v}_V=0
      \label{def:inner:item:nondege:2}
  \end{align}
  がわかる.
  \Cref{def:inner:item:nondege:1,def:inner:item:nondege:2}
  が成り立つことを,
  $\Braket{\bullet,\bullet}_V$が
  非退化
  であるという.

  \Cref{def:inner:item:nondege:1}は対偶をを取ると,
  \begin{align*}
    v\in V\setminus\Set{0_V}
    \iff
    \exists w\in V \text{ such that }
    \Braket{v,w}_V\neq 0
  \end{align*}
  となるのでこれが条件として書かれることもある.
  また,
  \Cref{def:inner:item:line:1:sca}
  から,
  \begin{align*}
    v=0_V
    \implies
    \forall w\in V,\Braket{v,w}_V=0
  \end{align*}
  が成り立つので,
  \begin{align}
    v\in V\setminus\Set{0_V}
    \implies
    \exists w\in V \text{ such that }
    \Braket{v,w}_V\neq 0
    \label{def:inner:item:nondege:1x}
  \end{align}
  のみが条件として書かれることもある.


  
  \Cref{def:inner:item:line}
    から, $\Braket{v_0,v}_V= \Braket{v_1,v}_V$ならば,
  \begin{align*}
    0=\Braket{v_0,v}_V-\Braket{v_1,v}_V
    =\Braket{v_0-v_1,v}_V
  \end{align*}
  となる.
  したがって,
  任意の$v\in V$に対して
  $\Braket{v_0,v}_V= \Braket{v_1,v}_V$ならば,
  \Cref{def:inner:item:nondege:1}から,
  $v_0-v_1=0_V$となり, $v_0=v_1$であることがわかる.
  逆は明らかであるから,
  \Cref{def:inner:item:line,def:inner:item:nondege:1}
  から
  \begin{align*}
    v_0=v_1\iff \forall v\in V, \Braket{v,v_0}_V= \Braket{v,v_1}_V
  \end{align*}
  がわかる.
\end{remark}
\begin{remark}
\Cref{def:inner:item:semi}を満たすことを,
  $\Braket{\bullet,\bullet}_V$は
半正定値(positive semidefinite)
  であるという.
  \Cref{def:inner:item:posi,def:inner:item:semi}を満たすことを,
  $\Braket{\bullet,\bullet}_V$は
  正定値(positive definite)
  であるという.

  \Cref{def:inner:item:line:1:sca}から,
  \begin{align*}
    v=0_V\implies \Braket{v,v}_V=0 
  \end{align*}
  はわかるので,
  \Cref{def:inner:item:posi}
  の条件の代わりに
  \begin{align*}
    \Braket{v,v}_V=0\implies v=0_V.
  \end{align*}
  と書かれることもある.
  この条件の対偶は,
  \begin{align*}
    v\in V\setminus\Set{0_V}
    \implies
    \Braket{v,v}_V\neq 0
  \end{align*}
  であるが,
  \Cref{def:inner:item:semi}を満たすときには,  
  \begin{align*}
    v\in V\setminus\Set{0_V}
    \implies
    \Braket{v,v}_V> 0
  \end{align*}
  と書くことができる.
  
  \Cref{def:inner:item:posi}を認めると,
  \Cref{def:inner:item:nondege:1x}の$w$として$v$自身がとれるので,
  \Cref{def:inner:item:nondege}
  が導かれる.
  そのため
  \Cref{def:inner:item:nondege}
  が条件として省かれることも多い.
\end{remark}

\begin{remark}
  内積とは,
  非退化正定値対称双線型形式のこと
  (形式というのは写像$V\times V\to \RR$を意味している).
\end{remark}

\begin{remark}
  複素ベクトルのときには,
  $\Braket{\bullet,\bullet}_V\colon V\times V\to \CC$
  として,
  \Cref{def:inner:item:sym}の条件を,
  \begin{align*}
    v,w\in V \implies \Braket{v,w}_V=\overline{\Braket{w,v}_V}
  \end{align*}
  に変更したものを考える.
  ただし$\overline{\bullet}$は複素共軛を表す.
  この条件を満たすことを,
  $\Braket{\bullet,\bullet}_V$
  はエルミート対称であるという.
\end{remark}

内積空間の構造とコンパチブルな写像を等長写像と呼ぶ.
\begin{definition}
  $(V,\Braket{}_V)$, $(W,\Braket{}_W)$を内積空間とし,
  $f\colon V\to W$を線型写像とする.
  以下の条件を満たすとき,
  $f$は等長写像であるという.
  \begin{align*}
    v,v'\in V\implies
    \Braket{v,v'}_V
    =
    \Braket{f(v),f(v')}_W
  \end{align*}
\end{definition}

\begin{remark}
  以下では, 文脈上誤解が生じないときには,
  $f$は
  $(V,\Braket{}_V)$から$(W,\Braket{}_W)$への
  等長写像
  であることを,
  単に$V$から$W$への等長写像であるという.
  また,
  本来であれば,
  等長写像$f\colon (V,\Braket{}_V) \to (W,\Braket{}_W)$
  などと書き表すべきであるが,
  等長写像$f\colon V\to W$などと書く.
\end{remark}

\begin{definition}
  $V$, $W$を内積空間とする.
  線型同型写像かつ等長写像である$f\colon V\to W$が存在するとき,
  $V$と$W$は内積空間として同型であるという.
\end{definition}
\begin{remark}
  $V$と$W$が内積空間として同型であるとき,
  $V$と$W$は内積空間として同じものであると思える.
\end{remark}
\begin{remark}
  等長写像の合成は等長写像である.
\end{remark}
\begin{remark}
  $V$を内積空間とする.
  $v\in V$に対し,
  \begin{align*}
    \|v\|_V = \sqrt{\Braket{v,v}_V}
  \end{align*}
  とおき, $\|\bullet\|_V$をノルムと呼ぶ.
  $\|v\|_V$は$v$の`長さ'に相当する.
  $f\colon V\to W$を等長写像とすると,
  \begin{align*}
    \|v\|_V &= \sqrt{\Braket{v,v}_V}\\
    &= \sqrt{\Braket{f(v),f(v)}_W}\\
    &= \|f(v)\|_W
  \end{align*}
  となり, ノルムが$f$によって変化しない.
\end{remark}
\begin{remark}
  $V$, $W$を内積空間とし,
  $f\colon V\to W$を等長写像とする.
  $f(v)=0_W$とする.
  このとき,
  $\|v\|_V=\|f(v)\|_W=\|0_W\|=0$
  であるので, $v=0_V$であることがわかる.
  したがって,
  $f$は単射である.
\end{remark}
\begin{definition}
  $V$を内積空間とする.
  $O(V)=\Set{f\in\Hom_\RR(V,V)\text{: 等長}}$
  とおく.
  $O(V)$を$V$上の直交群と呼ぶ.
  また,
  $f\in O(V)$を$V$上の直交変換と呼ぶ.
\end{definition}


  $V$を内積空間であるとする.
  $v_0\in V$に対し,
  \begin{align*}
    \beta_{v_0}\colon
    V&\to \RR\\
    v&\mapsto \Braket{v,v_0}_V
  \end{align*}
  とおく.
  このとき, (内積の定義\Cref{def:inner:item:line}より)
  $\beta_{v_0}$は線型写像である.
  つまり,
  \begin{align*}
    \beta_{v_0} \in \Hom_\RR(V,\RR) =V^\ast
  \end{align*}
  である.
  \begin{align*}
    \beta\colon
    V&\to V^\ast\\
    u&\mapsto \beta_{u}
  \end{align*}
  とおくと,
  (\Cref{def:inner:item:line:2:sum,def:inner:item:line:2:sca}より)
  $\beta$は線型写像である.
  (余談: $\beta$のことを内積を第2成分に関して
  カリー化した関数ということがある.)

  $V$は有限次元であるとする.
  $V$は有限次元であるとすると,
  $\beta$は同型写像である.
  つまり,
  \begin{align*}
    \forall \varphi \in V^\ast,
    \exists ! v_0 \in V \text{ s.t. }
    \beta_{v_0} = \varphi
  \end{align*}
  が成り立つ.
  これは, 言い換えると,
  \begin{align*}
    \forall \varphi \in V^\ast,
    \exists ! v_0 \in V \text{ s.t. }
    \forall v\in V, \varphi(v)=\Braket{v,v_0}_V
  \end{align*}
  となる.

\begin{remark}
  $V$が有限次元のとき,
  $\beta$は
  $V$と$V^\ast$の同型を与えるが,
  これは内積を取り替えると,
  異なる同型写像となる.
  内積を1つ決めるというのは,
  $V$と$V^\ast$の間の同型写像を1つ固定するということを意味する.
\end{remark}

$W$を有限次元内積空間とし,
$V$と同様に,
$w_0\in W$に対して,
\begin{align*}
  \beta_{w_0}\colon
  W&\to \RR\\
  w&\mapsto \Braket{w,w_0}_W
\end{align*}
とする.
$f\colon V\to W$を線型写像とする.
つまり, $f\in \Hom_\RR(V,W)$とする.
このとき
\begin{align*}
  \transposed{f}\colon
  W^\ast &\to V^\ast\\
  \varphi &\mapsto \varphi\circ f
\end{align*}
と定義され$\transposed{f}\in \Hom(W^\ast,V^\ast)$であった.

$w\in W$とする.
$\beta_w\in W^\ast$であるので,
$\transposed{f}(\beta_w)\in V^\ast$
である.
$V$が有限次元であるので,
$\beta\colon V\to V^\ast$は同型写像である.
したがって,
$\beta_{v_0}=\transposed{f}(\beta_w)$
を満たす$v_0\in V$が定まる.
この$v_0$を,
\begin{align*}
  f^\top(w) \in V
\end{align*}
とおく.
\begin{align*}
  f^\top\colon W&\to V\\
  w&\mapsto f^\top(w)
\end{align*}
という写像が定まる.
定義から,
\begin{align*}
  \transposed{f}(\beta_w)=\beta_{f^\top (w)}
\end{align*}
であるので,
$v\in V$, $w\in W$に対し,
\begin{align*}
  \Braket{f(v),w}_W
  &=\beta_w(f(v))\\
  &=(\beta_w \circ f)(v)\\
  &=(\transposed{f}(\beta_w))(v)\\
  &=\beta_{f^\top (w)}(v)\\
  &=\Braket{v,f^\top (w)}_V
\end{align*}
が成り立つ.
つまり,
\begin{align*}
  f^\top \colon W &\to V\\
  w&\mapsto f^\top (w)
\end{align*}
は
\begin{align*}
  v\in V, w\in W
  \implies
  \Braket{f(v),w}_W
  =\Braket{v,\beta_{f^\top (w)}}_V
\end{align*}
を満たす.
さらに,
$f^\top\in \Hom(W,V)$
であることもわかる.
$f^\top$も$f$の転置と呼ぶ.
\begin{remark}
  $f^\top \colon W \to V$
  は内積による$V$から$V^\ast$への同型$\beta$に依存している.
\end{remark}


\begin{remark}
  線型写像$f\colon V\to W$に対し,
  \begin{align*}
    \text{$f\colon V\to W$が等長写像}
    &\iff
    \forall v,v'\in V, \Braket{f(v),f(v')}_W=\Braket{v,v'}_V\\
    &\iff
    \forall v,v'\in V, \Braket{v,f^\top (f(v'))}_V=\Braket{v,v'}_V\\
    &\iff
    \forall v,v'\in V, \Braket{v,(f^\top \circ f)(v')}_V=\Braket{v,v'}_V\\
    &\iff
    \forall v'\in V, (f^\top \circ f)(v')=v'\\
    &\iff
    f^\top \circ f=\id_V
  \end{align*}
  である. $f\colon V\to V$について考えると,
  $f$が等長写像であることと$f^\top=f^{-1}$であることが
  同値であることがわかる.
\end{remark}


\begin{remark}
  $V$, $W$を有限次元ベクトル空間とすると,
  線型写像$f\colon V\to W$に対し,
  $f$の転置
  $\transposed{f}\in\Hom_\RR(W^\ast,V^\ast)$
  が定まる.
  この転置$\transposed{f}$は内積などによらずに定まる.
  しかし, $\transposed{f}$は$W$から$V$への写像ではない.
  一方$V$, $W$に内積があれば,
  $f\colon V\to W$の転置,
  $f^\top\in\Hom_\RR(W,V)$
  が定まる.
  この転置$f^\top$は$W$から$V$への写像ではあるものの,
  $f$に対して得られる転置$f^\top$は
  内積が変われば違うものとなる.

  すでに見たように,
  $V$, $W$に内積があれば,
  内積を通じて,
  $V$と$V^\ast$の同型$\beta$や,
  $W$と$W^\ast$の同型$\beta$が得られる.
  $f\in \Hom_\RR(V,W)$を一つ固定する.
  このとき$f^\top \in \Hom_\RR(W,V)$であるので,
  $w\in W$に対し, $f^\top(w)\in V$である.
  したがって, $\beta_{f^\top(w)}\in V^\ast$である.
  一方, $w\in W$であるので, $\beta_w\in W^\ast$である.
  $\transposed{f}\in \Hom_\RR(W^\ast,V^\ast)$であるので,
  $\transposed{f}(\beta_w)\in V^\ast$である.
  $V^\ast$の2つの元$\beta_{f^\top(w)}$と$\transposed{f}(\beta_w)$が得られたが,
  $v\in V$に対し,
  \begin{align*}
    \beta_{f^\top(w)}(v)
    &=\Braket{v,f^\top(w)}_V\\
    &=\Braket{f(v),w}_W\\
    (\transposed{f}(\beta_w))(v)
    &=(\beta_w\circ f)(v)\\
    &=\beta_w(f(v))\\
    &=\Braket{f(v),w}_W
  \end{align*}
  となるので,
  $\beta_{f^\top(w)}=\transposed{f}(\beta_w)$となる.
  つまり図式
  \begin{align*}
    \begin{CD}
      W @>f^\top>> V\\
      @V\beta VV @VV\beta V\\
      W^\ast @>>\transposed{f}> V^\ast
    \end{CD}
  \end{align*}
  が可換となっている.
\end{remark}


\section{(実)数ベクトル空間と行列}
