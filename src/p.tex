% !TeX root =./x2.tex
% !TeX program = pdfLaTeX
\chapter{準備}
ここでは, 線形代数の基本的な事項について復習する.
この原稿を通して使う基本的な用語や記号を定義することが主な目的であり,
必要最低限の準備をするため, 証明などは他の文献に任せる.
\section{線型写像に関する用語など}
本稿では,
特に断らない限り,
$V$は$n$次元ユークリッド空間を表すとし,
内積は $\Braket{\bullet,\bullet}$で表し,
ノルムは$\|\bullet \|$で表すとする.
つまり,
\begin{align*}
  V&=\RR^n\\
  &=\Set{\begin{pmatrix}\alpha_1\\\vdots\\\alpha_n\end{pmatrix} | \alpha_i \in \RR}.
\end{align*}
とする, ただし, $\RR$は実数全体のなす集合とする.
また, 
\begin{align*}
  \alpha=\begin{pmatrix}\alpha_1\\\vdots\\\alpha_n\end{pmatrix},\quad
  \beta=\begin{pmatrix}\beta_1\\\vdots\\\beta_n\end{pmatrix}\in V
\end{align*}
に対し,
\begin{align*}
\Braket{\alpha,\beta}=\sum_{i=1}^n \alpha_i \beta_i
\end{align*}
とし,
\begin{align*}
  \|\alpha \|=\sqrt{\Braket{\alpha,\alpha}}
\end{align*}
とする.


$\zzero$は$V$の零ベクトルを表し,
$\Set{\ee_1,\ldots,\ee_n}$は$V$の標準基底であるとする.
つまり,
\begin{align*}
 \zzero&=\begin{pmatrix}0\\\vdots\\0\end{pmatrix},\\
 \ee_1&=\begin{pmatrix}1\\0\\\vdots\\0\end{pmatrix},
 \ee_2=\begin{pmatrix}0\\1\\\vdots\\0\end{pmatrix},
 \ldots,
 \ee_n=\begin{pmatrix}0\\\vdots\\0\\1\end{pmatrix}
\end{align*}
とする.

$E_n$で単位行列.
$E_{k,l}$で行列単位.
$\diag$で対角行列.



直和分解, 写像の直和


$f\circ g$を$fg$とかく.

$f\colon V\to V$を$V$上の変換とよぶ.

直交行列.

固有値固有ベクトル.
直交行列の固有値は1,-1
固有空間は直交する.

単位行列
対角行列
行列単位

\section{群について}
対称群, 互換, 隣接互換
群の定義
同型
$\ZZ^{\times}$ $(\ZZ^{\times})^n$ 生成系
生成系
位数.

\chapter{鏡映}

\section{鏡映の定義と例}


$\alpha \in V\setminus\Set{\zzero}$に対し,
\begin{align*}
  H_\alpha = \Set{\beta \in V | \Braket{\alpha,\beta}=0}
\end{align*}
とおき, $\alpha$を法ベクトルとする(中心的な)超平面と呼ぶ.
また,
\begin{align*}
  \RR\alpha = \Set{c \alpha | c\in \RR}
\end{align*}
とおく. $\RR\alpha$は原点を通り$\alpha$を方向ベクトルとする直線である.
このとき, $V$は,
\begin{align*}
  V =   H_\alpha \oplus   \RR\alpha
\end{align*}
と(内部)直和に分解される.  つまり, 次の2つを満たす:
\begin{align*}
  &V = \Set{ \beta + \gamma | \beta \in H_\alpha, \gamma \in \RR\alpha},\\
  &H_\alpha \cap   \RR\alpha =\Set{\zzero}.
\end{align*}

\begin{definition}
  $\alpha \in V\setminus\Set{\zzero}$とし,
  $f\colon V\to V$は線型写像であるとする.
  次の2つの条件を満たすとき,
  $f$が$\alpha$に関する鏡映であるという:
  \begin{enumerate}
  \item $f(\alpha)=-\alpha$.
  \item $\beta \in H_\alpha \implies f(\alpha)=\alpha$.    
  \end{enumerate}
  $f$が$\alpha$に関する鏡映であるような$\alpha\in V\setminus\Set{\zzero}$が存在するとき,
  $f$は鏡映であるという.
\end{definition}
$n$次正方行列は,
$\alpha\in V$に対し,
$A\alpha\in V$を対応させる$V$から$V$への線型写像であると思うことができる.
対応する線型写像が鏡映であるとき, その行列も鏡映と呼ぶことにする.
行列の言葉に定義を書き直すと以下のようになる.
\begin{definition}
  $\alpha \in V\setminus\Set{\zzero}$とし,
  $A$は$n$次正方行列であるとする.
  次の2つの条件を満たすとき,
  $A$が$\alpha$に関する鏡映であるという:
  \begin{enumerate}
  \item $A\alpha=-\alpha$.
  \item $\beta \in H_\alpha \implies A\alpha=\alpha$.    
  \end{enumerate}
\end{definition}

\begin{example}
  \begin{align*}
    P_0=
    \begin{pmatrix}
      1 & 0 \\
      0 & -1 
    \end{pmatrix}.
  \end{align*}
  とし,
  $\alpha=\ee_2$とおくと,
  $P_0$は$\alpha$に関する鏡映である.
実際, 
\begin{align*}
  P_0\alpha = 
      \begin{pmatrix}
      1 & 0 \\
      0 & -1 
      \end{pmatrix}
      \alpha
    =
      \begin{pmatrix}
      1 & 0 \\
      0 & -1 
    \end{pmatrix}
    \begin{pmatrix}
      0  \\
      1  
    \end{pmatrix}
    =
    \begin{pmatrix}
      0  \\
      -1  
    \end{pmatrix}
    =
    -
    \begin{pmatrix}
      0  \\
      1  
    \end{pmatrix}
    =
    -\alpha
\end{align*}
である.  また,
\begin{align*}
  H_\alpha = \Set{c \ee_1 | c\in \RR}
\end{align*}
であるが,
\begin{align*}
    \begin{pmatrix}
      1 & 0 \\
      0 & -1 
    \end{pmatrix}
    \begin{pmatrix}
      c  \\
      0  
    \end{pmatrix}
    =
    \begin{pmatrix}
      c  \\
      0  
    \end{pmatrix}
\end{align*}
である.
\end{example}

\begin{example}
  $P_0$は鏡映であった.
  もっと一般に,
  実数$\theta$に対し,
\begin{align*}
  P_{2\theta}=
  \begin{pmatrix}
    \cos(2\theta) & \sin(2\theta) \\
    \sin(2\theta) & -\cos(2\theta) 
  \end{pmatrix},\\
  \alpha_\theta
  =
  \begin{pmatrix}
    \sin(\theta) \\
    \cos(\theta) 
  \end{pmatrix}
\end{align*}
とおけば, $P_{2\theta}$は$\alpha_\theta$に関する鏡映である.
実際,
\begin{align*}
  H_{\alpha_{\theta}} = \Set{c \alpha_{\theta+\frac{\pi}{2}}|c\in \RR} 
\end{align*}
であるので,
\begin{align*}
  P_{2\theta}\alpha_{\theta} &= - \alpha_{\theta},\\
  P_{2\theta}(c\alpha_{\theta+\frac{\pi}{2}}) &= c\alpha_{\theta+\frac{\pi}{2}}
\end{align*}
を示せばよいが, これらは直接, 計算することで確かめることができる.
\end{example}


\begin{example}
  $P_0$は鏡映であった.
  もっと一般に,
  $F_k$を$(k,k)$成分は$-1$, 他の対角成分は$1$であるような$n$次対角行列とする.
 $E_{k,l}$で$(k,l)$行列単位, つまり, $(k,l)$成分のみ1で他は0であるような行列とすると,
  \begin{align*}
    F_k = \diag(1,\ldots,1,-1,1,\ldots,1)=E_n-2E_{k,k}
  \end{align*}
  である.
  $\alpha=\ee_k$とすると, $F_k$は$\alpha$ に関する鏡映である.
  $F_k \alpha = -\alpha$となることは, 直接計算すればわかる.
  また, $\beta\in V$に対し,
  \begin{align*}
    \Braket{\alpha,\beta}= \beta_k 
  \end{align*}
  であるので,
  \begin{align*}
    H_\alpha = \Set{
      \beta=
      \begin{pmatrix}
        \beta_1\\
        \vdots\\
        \beta_n
      \end{pmatrix}
      | \beta_k=0}
  \end{align*}
  である.  $\beta\in H_\alpha$に対し, 
  $F_k \beta = -\beta$となることは, 直接計算すればわかる.
\end{example}

\begin{example}
  $T_{k,l}$を$(k,k)$成分と$(l,l)$成分は$0$, 他の対角成分は$1$,
$(k,l)$成分と$(l,k)$成分は$1$, 他の成分は$0$
  であるような$n$次正方行列とする.
  つまり,
  \begin{align*}
    T_{k,l} =E_n-E_{k,k}-E_{l,l}+E_{l,k}+E_{k,l}
  \end{align*}
  である.
  $\alpha=\ee_k-\ee_l$とすると, $T_{k,l}$は$\alpha$ に関する鏡映である.
  実際,
  \begin{align*}
    T_{k,l} \alpha = T_{k,l}(e_k-e_l) = T_{k,l}e_k-T_{k,l}e_l = e_l - e_k = -(e_k - e_l) = -\alpha 
  \end{align*}
  である.  
  また, $\beta\in V$に対し,
  \begin{align*}
    \Braket{\alpha,\beta}= \beta_k - \beta_l
  \end{align*}
  であるので,
  \begin{align*}
    H_\alpha = \Set{
      \beta=
      \begin{pmatrix}
        \beta_1\\
        \vdots\\
        \beta_n
      \end{pmatrix}
      | \beta_k=\beta_l}
  \end{align*}
  である.
  $\beta\in H_\alpha$に対し, 
  $F_k \beta = -\beta$となることは, 直接計算すればわかる.
\end{example}



\begin{example}
  $k\neq l$に対し,
  $F_{k,l}=F_kF_lT_{k,l}$とおく.
  つまり$F_{k,l}$は$(k,k)$成分と$(l,l)$成分は$0$, 他の対角成分は$1$,
  $(k,l)$成分と$(l,k)$成分は$-1$, 他の成分は$0$であるような$n$次正方行列であり,
  \begin{align*}
    F_{k,l} = E_n-E_{k,k}-E_{l,l}-E_{k,l}-E_{l,k}
  \end{align*}
  である.
  $\alpha=\ee_k+\ee_l$とすると, $F_{k,l}$は$\alpha$ に関する鏡映である.
  実際,
  \begin{align*}
    F_{k,l} \alpha &= F_kF_lT_{k,l}(\ee_k+\ee_l)\\
    &= F_kF_l(\ee_k+\ee_l)\\
    &= F_kF_l\ee_k+F_kF_l\ee_l\\
    &= F_k\ee_k-F_k\ee_l\\
    &= -\ee_k-\ee_l\\
    &= -(\ee_k+\ee_l)\\
    &= -\alpha
  \end{align*}
  となる.
  また, $\beta\in V$に対し,
  \begin{align*}
    \Braket{\alpha,\beta}= -\beta_k-\beta_l 
  \end{align*}
  であるので,
  \begin{align*}
    H_\alpha = \Set{
      \beta=
      \begin{pmatrix}
        \beta_1\\
        \vdots\\
        \beta_n
      \end{pmatrix}
      | \beta_k=-\beta_l}
  \end{align*}
  である.  $\beta\in H_\alpha$に対し, 
  $F_k \beta = -\beta$となることは, 直接計算すればわかる.
\end{example}

\begin{example}
  $\alpha\in V\setminus\Set{\zzero}$とする.
  $\beta\in V$に対して,
  \begin{align*}
    s_\alpha (\beta) = \beta - 2\frac{\Braket{\alpha,\beta}}{\Braket{\alpha,\alpha}}\alpha
  \end{align*}
  と定義する. $s_\alpha$は$V$から$V$への線型写像である.
  また, $s_\alpha$は$\alpha$に関する鏡映である.
  実際
  \begin{align*}
    s_\alpha (\alpha) = \alpha - 2\frac{\Braket{\alpha,\alpha}}{\Braket{\alpha,\alpha}}\alpha
      = \alpha - 2\alpha=-\alpha
  \end{align*}
であり, $\beta\in H_\alpha$に対しては,  
  \begin{align*}
    s_\alpha (\beta) = \beta - 2\frac{\Braket{\alpha,\beta}}{\Braket{\alpha,\alpha}}\alpha
    = \beta - 0\alpha =\beta
  \end{align*}
  となる,
\end{example}

\section{鏡映の性質}
ここでは, $V$から$V$への線型写像で鏡映であるものの性質について考える.
\begin{lemma}
  \label{lemma:reflisunique}
  $s$も$s'$も$\alpha$に関する鏡映であるとする.
  このとき, $s=s'$.
\end{lemma}
\begin{proof}
  $V =   H_\alpha +  \RR\alpha$
  であるので,
  $s(\alpha)=s'(\alpha)$と,
  $\beta\in H_\alpha$に対して
  $s(\beta)=s'(\beta)$を示せば十分である.
  これらは,
  $\alpha$に関する鏡映の定義から直接わかる.
\end{proof}
\Cref{lemma:reflisunique}を使うと次がすぐわかる.
\begin{prop}
  $s$を$V$上の線形変換であるとする.
  $s$が$\alpha$に関する鏡映であるとき, $s=s_\alpha$.
\end{prop}

\begin{prop}
  $\alpha, \alpha' \in V\setminus\Set{\zzero}$とする.
  このとき, 次は同値:
  \begin{enumerate}
  \item $s_\alpha=s_{\alpha'}$.
  \item $\alpha=c\alpha'$となる$c\in \RR$が存在する.
  \end{enumerate}
\end{prop}
\begin{proof}
  $s_\alpha=s_{\alpha'}$とする.
  このとき,
  \begin{align*}
    s_\alpha(\alpha) &=-\alpha,\\
    s_{\alpha'} (\alpha) &= \alpha - 2\frac{\Braket{\alpha',\alpha}}{\Braket{\alpha',\alpha'}}\alpha'
  \end{align*}
  であるので,
  \begin{align*}
    -\alpha &= \alpha - 2\frac{\Braket{\alpha',\alpha}}{\Braket{\alpha',\alpha'}}\alpha'\\
    \alpha &= \frac{\Braket{\alpha',\alpha}}{\Braket{\alpha',\alpha'}}\alpha'
  \end{align*}
  となる.

  $\alpha=c\alpha'$とする.
  このとき,
  \begin{align*}
    s_{\alpha}(\alpha')=s_{\alpha}(c\alpha)=cs_{\alpha}(\alpha)=-c\alpha=-\alpha'
  \end{align*}
  である.
  また,
  $\alpha\neq \zzero$
  であるので, $c\neq 0$である.
  $\beta \in H_{\alpha'}$とすると,
  \begin{align*}
    \Braket{\alpha,\beta}=\Braket{\frac{1}{c}\alpha',\beta}=\frac{1}{c}\Braket{\alpha',\beta}=0
  \end{align*}
  であるので, $\beta \in H_\alpha$である.
  したがって,
  $s_\alpha(\beta)=\beta$である.
  よって, $s_\alpha$は$\alpha'$に関する鏡映である.
  \Cref{lemma:reflisunique}
  より, $s_\alpha=s_{\alpha'}$である.
\end{proof}


\begin{prop}
  $s$を鏡映とする.
  このとき, $s^2=\id_V$.
\end{prop}
\begin{proof}
  $s$を$\alpha$に関する鏡映とする.
  $s^2(\alpha)=s(-\alpha)=-s(\alpha)=-(-\alpha)=\alpha$
  である.
  また, $\beta\in V$に対して,
  $s^2(\beta)=s(\beta)=\beta$である.
  よって,
  $s$は$V$上の恒等写像である.
\end{proof}

\begin{cor}
  $s$を鏡映とする.
  このとき, $s=s^{-1}$.
\end{cor}


\begin{prop}
  $s$を鏡映とする.
  このとき, $s$は内積を保存する.
  つまり, $\lambda,\mu\in V$に対して, $\Braket{s(\lambda),s(\mu)}=\Braket{\lambda,\mu}$である.
\end{prop}
\begin{proof}
  $s$を$\alpha$に関する鏡映とする.
  $\lambda,\mu\in V$とすると,
  $V=H_\alpha+\RR\alpha$であるので,
  \begin{align*}
    \lambda &= \beta_\lambda + a_\lambda \alpha, 
    &\mu &= \beta_\mu + a_\mu \alpha 
  \end{align*}
  をみたす$\beta_\lambda, \beta_\mu \in H_\alpha$と$a_\lambda,a_\mu \in \RR$が存在する.
  \begin{align*}
    \Braket{s(\lambda),s(\mu)}&=
    \Braket{s(\beta_\lambda + a_\lambda \alpha), s(\beta_\mu + a_\mu \alpha) }\\
    &=
    \Braket{\beta_\lambda - a_\lambda \alpha, \beta_\mu - a_\mu \alpha }\\
    &=
    \Braket{\beta_\lambda , \beta_\mu  }
    +\Braket{\beta_\lambda ,  - a_\mu \alpha }
    +\Braket{ - a_\lambda \alpha, \beta_\mu }
    +\Braket{ - a_\lambda \alpha, - a_\mu \alpha }\\
    &=
    \Braket{\beta_\lambda , \beta_\mu  }
    +0
    +0
    +\Braket{ a_\lambda \alpha, a_\mu \alpha }\\
    &=
    \Braket{\beta_\lambda , \beta_\mu  }
    +\Braket{ a_\lambda \alpha, a_\mu \alpha }\\
    \Braket{\lambda,\mu}&=
    \Braket{\beta_\lambda + a_\lambda \alpha, \beta_\mu + a_\mu \alpha }\\
    &=
    \Braket{\beta_\lambda , \beta_\mu  }
    +\Braket{\beta_\lambda ,  a_\mu \alpha }
    +\Braket{a_\lambda \alpha, \beta_\mu }
    +\Braket{a_\lambda \alpha, a_\mu \alpha }\\
    &=
    \Braket{\beta_\lambda , \beta_\mu  }
    +0
    +0
    +\Braket{ a_\lambda \alpha, a_\mu \alpha }\\
    &=
    \Braket{\beta_\lambda , \beta_\mu  }
    +\Braket{ a_\lambda \alpha, a_\mu \alpha }
  \end{align*}
  がなりたつ. 
\end{proof}

\begin{prop}
  $t$は内積を保つ線型写像であるとする.
  $\alpha\in V\setminus\Set{\zzero}$に対し,
  $ts_\alpha t^{-1}=s_{t(\alpha)}$.
\end{prop}
\begin{proof}
\Cref{lemma:reflisunique}より,
  $ts_\alpha t^{-1}$が$t(\alpha)$に関する鏡映であることを示せば十分である.
\begin{align*}
  ts_\alpha t^{-1}(t(\alpha))=ts_\alpha(\alpha)=t(-\alpha)=-t(\alpha)
\end{align*}
である. また$\beta\in H_{t(\alpha)}$に対して,
\begin{align*}
\Braket{\alpha,t^{-1}(\beta)}
=\Braket{t(\alpha),\beta}
=0
\end{align*}
であるので, $t^{-1}(\beta)\in H_\alpha$である.
したがって,
\begin{align*}
  ts_\alpha t^{-1}(\beta)=ts_\alpha(t^{-1}(\beta))=t(t^{-1}(\beta))=\beta.
\end{align*}
\end{proof}

\begin{prop}
  直交変換$s$に対し以下は同値:
  \begin{enumerate}
  \item $s$は鏡映である.
  \item 固有値$-1$に属する固有空間の次元は$1$次元である.
  \end{enumerate}
\end{prop}
\begin{proof}
  $s$が$\alpha$に関する鏡映であるとする.
  このとき, 定義から,
  固有値$-1$に属する固有空間は$\RR\alpha$であり,
  固有値$1$に属する固有空間は$H_\alpha$である.
  よって固有値$-1$に属する固有空間の次元は$1$次元である.

  固有値$-1$に属する固有空間の次元は$1$次元であるとする.
  このとき, 固有値$-1$に属する固有空間は$\RR\alpha$とかける.
  直交変換の固有値は, $1$か$-1$であり, 固有空間は互いに直交するので,
  固有値$1$に属する固有空間は$H_\alpha$である.
  固有値の定義から,
  $s(\alpha)=\alpha$と
  $\beta\in H_\alpha$に対して$s(\beta)=\beta$である.
  したがって,
  $s$は$\alpha$に関する鏡映である.
\end{proof}

鏡映変換に関する性質を, 行列の言葉に書き直しておく.
\begin{cor}
  $A$は$n$次正方行列とし, 鏡映であるとする.
  このとき,
  \begin{align*}
    A^2&=E_n,\\
    \transposed{A}A&=E_n,\\
    A&=A^{-1}=\transposed{A},\\
    \exists P &\text{ such that } PAP^{-1}=\diag(1,\ldots,1,-1)\\
    \det(A)&=-1.
  \end{align*}
\end{cor}



\begin{example}
  $E_n$は鏡映ではない.
  $\alpha\in V$に対し,
  \begin{align*}
    E_n\alpha=-\alpha
  \end{align*}
  とすると,
  \begin{align*}
    \alpha=\zzero
  \end{align*}
  となり, $V\setminus\Set{\zzero}$の元では,
  $E_n\alpha=-\alpha$を満たすことがないので,
  $E_n$は鏡映ではない.
\end{example}

\begin{example}
 実数$\theta$に対し,
  \begin{align*}
    R_\theta=
    \begin{pmatrix}
      \cos(\theta) & -\sin(\theta) \\
      \sin(\theta) & \cos(\theta)
    \end{pmatrix}.
  \end{align*}
 とおくと, これは鏡映ではない.
 \begin{align*}
    R_\theta \alpha = -\alpha
 \end{align*}
 とする. このとき,
 \begin{align*}
    (R_\theta + E_2)\alpha = \zzero
 \end{align*}
 である.
 \begin{align*}
   \det(R_\theta + E_2) &= 
    \det \begin{pmatrix}
      \cos(\theta) +1 & -\sin(\theta) \\
      \sin(\theta) & \cos(\theta) +1 
    \end{pmatrix}\\
    &=(\cos(\theta) +1)^2 + (\sin(\theta))^2\\
    &=2(1+\cos(\theta))
 \end{align*}
 である.
 したがって, $\cos(\theta)\neq 0$のときには,
 $R_\theta + E_2$は正則であるから, $\alpha=\zzero$のみが解となり,
 この場合は鏡映ではないことがわかる.
 また,  $\cos(\theta)= 0$のときには, $R_\theta=E_2$であり,
 この場合も鏡映ではない.
\end{example}

\begin{example}
  $\sigma \in S_n$に対し,
  $(\sigma(1),1)$成分,\ldots
  $(\sigma(n),n)$成分
  は$1$で. 他の成分は$0$であるような$n$次正方行列を$A_\sigma$とおく.
  つまり,
  \begin{align*}
    A_\sigma = \sum_{j=1}^{n} E_{\sigma(j),j} = \sum_{i=1}^{n} E_{i,\sigma^{-1}(i)}
  \end{align*}
  である.
  $\sigma,\tau\in S_n$に対し, $A_\sigma=A_\tau$であることと,
  $\sigma=\tau$であることは同値である.
  直接計算することで,
  $A_\sigma \ee_k=\ee_{\sigma(k)}$であることがわかる.
  $\sigma\in S_n$に対し次は同値である:
  \begin{enumerate}
  \item $\sigma$が互換である.
  \item $A_\sigma$が鏡映である.
  \end{enumerate}
  $T_{i,j}$が互換であることは示した.
  逆を示す.
  $\sigma$が互換ではないとする.
  $\sigma=\varepsilon$のとき, $A_\sigma=E_n$であるので, これは鏡映ではない.
  $\sigma\neq\varepsilon$とする.
  このとき$\sigma(i_1)\neq i_1$を満たす$i_1$が存在する.
  $\sigma(i_1)=i_2$とおき,
  $\alpha=\ee_{i_1}-\ee_{i_2}$とする.
  このとき,
  \begin{align*}
    A_\sigma A_\sigma\alpha
    &= A_\sigma A_\sigma (\ee_{i_1}-\ee_{i_2})\\
    &= A_\sigma(\ee{\sigma(i_1)}-\ee{\sigma(i_2)})\\
    &= \ee{\sigma\sigma(i_1)}-\ee{\sigma\sigma(i_2)}
  \end{align*}
  である. 鏡映は二乗すると単位行列となるので,
  $\ee{\sigma\sigma(i_1)}-\ee{\sigma\sigma(i_1)}\neq \alpha$ならば,
  $A_\sigma$は鏡映ではない.
  $\ee{\sigma\sigma(i_1)}-\ee{\sigma\sigma(i_2)}=\ee{i_1}-\ee{i_2}$と仮定する.
  このとき, $i_2=i_1$である.
  したがって,  $A_\sigma \alpha = -\alpha$である.
  ここで,
  $i\in \Set{i_1, i_2}$の他に$\sigma(i)=i$となる$i$がなければ, $\sigma$は互換である.
  $j_1\not\in \Set{i_1,i_2}$が
  $\sigma(j_1)\neq j_1$を満たすとする.
  $\sigma(j_1)=j_2$とし, $\beta=\ee_{j_1}-\ee_{j_2}$とおく.
  $\alpha$のときと同じ議論で,
  $\sigma(j_2)\neq j_1$であるときには, $A_\sigma$は鏡映ではないことがわかる.
  $\sigma(j_2)= j_1$を仮定すると, $A_\sigma \beta=-\beta$がなりたつ.
  $\Set{i_1,i_2}\neq \Set{j_1,j_2}$であるので,
  $\alpha$と$\beta$は一次独立である.
  したがって, $A_\sigma$の固有値$1$に属する固有空間の次元は$1$ではないので,
  鏡映ではない.
\end{example}


\chapter{有限鏡映群}

\section{有限鏡映の定義}
\begin{definition}
  次の条件をみたすとき, 
  $W$を鏡映群と呼ぶ:
  \begin{enumerate}
  \item
    $W=\Braket{s_1,\ldots,s_t}$をみたす
    鏡映変換
    $s_1,\ldots,s_t$が存在する.
  \end{enumerate}
\end{definition}
また行列に対しても同様に定義をする.
\begin{definition}
  次の条件をみたすとき, 
  $W$を有限鏡映群と呼ぶ:
  \begin{enumerate}
  \item
    $W=\Braket{A_1,\ldots,A_t}$をみたす
    鏡映
    $A_1,\ldots,A_t$が存在する.
  \end{enumerate}
\end{definition}
鏡映群$W$が有限集合であるとき,
$W$を有限鏡映群と呼ぶ.

\begin{definition}
  $W$を$V$上の鏡映群とする.
  次の条件をみたすとき, 
  $W$は本質的であるという:
  \begin{align*}
    \forall \alpha\in V\setminus\Set{\zzero},\ 
    \exists w\in W
    \text{ s.t. }
    w(\alpha)\neq \alpha.
  \end{align*}
\end{definition}

\section{有限鏡映群の例}
\subsection{$I_2(m)$}
$m$を正の整数とする.
ここでは, 記号の簡略化のために
\begin{align*}
  P_i&=P_{\frac{2\pi}{m}}\\
  R_i&=R_{\frac{2\pi}{m}}
\end{align*}
とおく.
 まず, $P_i$, $R_i$の計算規則について調べておく.
\begin{lemma}
  \begin{align*}
    P_i &= P_{i+m}\\
    R_i &= R_{i+m}
  \end{align*}
\end{lemma}
\begin{lemma}
  \label{lem:i2m:rel:ppr}
  \begin{align*}
    P_i P_j&= R_{j-i}.
  \end{align*}
\end{lemma}
\begin{lemma}
  \begin{align*}
    R_i R_j&= R_{i+j}
  \end{align*}
\end{lemma}
\begin{cor}
  \label{lem:i2m:fomula:prp}
  \begin{align*}
    P_i R_j&= P_{i+j}
  \end{align*}
\end{cor}

\begin{cor}
  \label{lem:i2m:formula:conjP}
  \begin{align*}
    P_i P_j P_i&= P_iR_{i-j}=P_{2i-j}.
  \end{align*}
\end{cor}

\begin{cor}
  \label{lem:i2m:formula:conjR}
  \begin{align*}
    R_{j}P_iR_{-j}=P_{i-2j}.
  \end{align*}
\end{cor}
\begin{proof}
  \Cref{lem:i2m:rel:ppr}より,
  \begin{align*}
    P_{i}P_{j+i}&=R_{j}\\
  \end{align*}
であるので,
  \begin{align*}
    R_{j}P_iR_{-j}&=P_{i}P_{j+i}P_{i}R_{-j}\\
  \end{align*}
  \Cref{lem:i2m:fomula:prp,lem:i2m:formula:ppr}より,
\begin{align*}
  P_{i}P_{j+i}P_{i}R_{-j}&=P_{i}P_{j+i}P_{i-j}\\
    &=P_{i}R_{i-j-j-i}\\
    &=P_{i}R_{-2j}\\
    &=P_{i-2j}.
  \end{align*}
\end{proof}

    

\begin{lemma}
  \label{lem:i2m:conj}
  \begin{align*}
    P_i=(P_{i+1} P_{i+2}\cdots P_{i+2n-1}) P_{i+2n}(P_{i+1} P_{i+2}\cdots P_{i+2n-1})^{-1}
  \end{align*}
\end{lemma}
\begin{proof}
  \Cref{lem:i2m:rel:ppr}
  より,
  \begin{align*}
    R_{1}=P_{1}P_{2}=P_{2}P_{3}=\cdots =P_{m-1}P_{m}
  \end{align*}
  である.  したがって,
  \begin{align*}
    P_{i}P_{i+1}=P_{i+1}P_{i+2}
  \end{align*}
  であるので,
  \begin{align*}
    P_{i}=P_{i+1}P_{i+2}P_{i+1}^{-1}
  \end{align*}  
となる. 同様に
  \begin{align*}
    P_{i}&=P_{i+1}P_{i+2}P_{i+1}^{-1}\\
    &=P_{i+1}(P_{i+2}P_{i+3}P_{i+2}^{-1})P_{i+1}^{-1}.
  \end{align*}
  を得ることができる. これを繰り返すことで, 欲しい等式が得られる.
\end{proof}
$D_{2\cdot m}$を次で定義する:
\begin{align*}
  D_{2\cdot m} = \Set{P_i,R_i|i\in \ZZ} = \Set{P_i,R_i|i=0,\ldots,m-1}.
\end{align*}
計算規則から,
\begin{align*}
   D_{2\cdot m} = \Braket{P_0,\ldots,P_{m-1}}
\end{align*}
であることがわかる.
$P_i$は鏡映であったので,
$D_{2\cdot m}$は有限鏡映群である.
$D_{2\cdot m}$は位数$2m$の二面体群とよばれる.

\begin{prop}
$m>1$のとき, $D_{2\cdot m}$は本質的である.
\end{prop}
\begin{proof}
  \begin{align*}
    \alpha=\begin{pmatrix}
      \alpha_1 \\
      \alpha_2
    \end{pmatrix}\in V\setminus\Set{\zzero}
  \end{align*}
  とする.
  $\alpha_2\neq 0$のとき,
  \begin{align*}
    P_0 \alpha =
    \begin{pmatrix}
      1&0\\
      0&-1
    \end{pmatrix}  
    \begin{pmatrix}
      \alpha_1\\
      \alpha_2
    \end{pmatrix}
    =
    \begin{pmatrix}
      \alpha_1\\
      -\alpha_2
    \end{pmatrix}
    \neq \alpha.
  \end{align*}
  また$\alpha_2-0$のとき,
  $\alpha\neq \zzero$であることから,
  $\alpha_1\neq 0$ となる.
  このとき,
  \begin{align*}
    P_1 \alpha =
    \begin{pmatrix}
      \cos(\frac{2\pi}{m}&\sin(\frac{2\pi}{m}\\
      \sin(\frac{2\pi}{m}&-\cos(\frac{2\pi}{m}
    \end{pmatrix}  
    \begin{pmatrix}
      \alpha_1\\
      0
    \end{pmatrix}
    =
    \begin{pmatrix}
      \alpha_1\cos(\frac{2\pi}{m}\\
      \alpha_1\sin(\frac{2\pi}{m}
    \end{pmatrix}  
    \neq
    \begin{pmatrix}
      \alpha_1\\
      0
    \end{pmatrix}
    .
  \end{align*}
\end{proof}
\begin{prop}
  $m=1$のとき, つまり $D_{2\cdot 1}$は本質的ではない.
\end{prop}

\begin{proof}
  $D_{2\cdot 1}=\Set{P_0,R_0}$である.
  \begin{align*}
    \alpha=
    \begin{pmatrix}
      1\\0
      \end{pmatrix}
  \end{align*}
  とすると,
  $P_0\alpha = R_0\alpha= \alpha $である.
\end{proof}

\begin{lemma}
  \label{lem:i2m:modd:eq}
  $m$を奇数とする.
  $A\in D_{2\cdot m} \setminus \Set{E_2}$に対し,
  次は同値:
  \begin{enumerate}
  \item $A$は鏡映である.
  \item $A^2=E_2$.
  \end{enumerate}
\end{lemma}
\begin{proof}
  $A$が鏡映である $A^2=E_2$を満たすことは既に見た.
  逆を示す.
  $A\in D_{2\cdot m}\setminus \Set{E_2}$は鏡映ではないとする.
  つまり, $A=R_{k}$とかけるとする.
  ただし, $A\neq E_2$であるので,
  $0<k <m$である.
  $A^2=R_{2k}$であるが, $m$は奇数であるから,
  $2k$を$m$でわった余りが$0$となることはない.
  つまり, $A^{2}\neq E_2$である.
\end{proof}

\begin{prop}
  $m$が奇数であるとする.
  このとき,
  \begin{align*}
    \Set{P_i | i \in \ZZ} =
    \Set{AP_0A^{-1} | A\in D_{2\cdot m}}.
  \end{align*}
  つまり, $\Set{P_i | i \in \ZZ}$は$P_0$を含む共軛類である.
\end{prop}
\begin{proof}
\Cref{lem:i2m:conj}  
 から 
 $P_i\in \Set{AP_0A^{-1} | A\in D_{2\cdot m}}$はすぐわかる.

 $A\in D_{2\cdot m}$とする.
 このとき,
 \begin{align*}
   (AP_0A^{-1})^2 =  AP_0A^{-1} AP_0A^{-1}
   =  AP_0P_0A^{-1}
       =  AA^{-1}
       =  E_2
 \end{align*}
であるので, \cref{lem:i2m:modd:eq}より,
$AP_0A^{-1}$は鏡映である.
したがって,
$AP_0A^{-1}\in \Set{P_i | i \in \ZZ}$.
\end{proof}

\begin{prop}
  $m$を偶数とする.
  このとき,
  \begin{align*}
    \Set{P_{2i}|i\in\ZZ}&=\Set{AP_0A^{-1} | A \in D_{2\cdot m}}\\
    \Set{P_{2i+1}|i\in\ZZ}&=\Set{AP_1A^{-1} | A \in D_{2\cdot m}}.
  \end{align*}
つまり, 
$\Set{P_{2i}|i\in\ZZ}$ は $P_0$を含む共軛類,
$\Set{P_{2i+1}|i\in\ZZ}$は$P_1$を含む共軛類である.
\end{prop}
\begin{proof}
  \Cref{lem:i2m:formula:conjP,lem:i2m:formula:conjR}
  より, $AP_iA^{-1}=P_j$であるなら, $i$と$j$が等しいことがわかる.
  したがって,
  \begin{align*}
    \Set{AP_0A^{-1} | A \in D_{2\cdot m}} \cap
    \Set{AP_1A^{-1} | A \in D_{2\cdot m}} = \emptyset
  \end{align*}
  となることがわかる.

  また,
  \cref{lem:i2m:conj}  
  から,
  $P_{2i}\in \Set{AP_0A^{-1} | A \in D_{2\cdot m}}$,
  $P_{2i+1}\in \Set{AP_1A^{-1} | A \in D_{2\cdot m}}$がわかる.
\end{proof}

\subsection{$A_{n-1}$}

$W=\Set{A_\sigma|\sigma\in S_n}$とおく.
$\sigma,\tau \in S_n$に対し,
$A_\sigma A_\tau=A_{\sigma\tau}$
が成り立つことは直接計算することで確かめられる.
また,
$A_{\varepsilon}=E_n$であり,
$A_\sigma^{-1}=A_{\sigma^{-1}}$である.
したがって, $W$は群である.
さらに,
$f(\sigma)=A_\sigma$とおくと,
$f$は$S_n$から$W$への全単射である.
さらに, これは群準同型写像であるので,
$S_n$と$W$は同型である.


$\tau_{i,j}\in S_n$を$i$と$j$を入れ替える互換とする.
つまり,
\begin{align*}
  \tau_{i,j}(k)=
  \begin{cases}
    j&(k=i)\\
    i&(k=j)\\
    k&(k\not\in\Set{i,j})
  \end{cases}
\end{align*}
とする.
このとき, $A_{\tau_{i,j}}=T_{k,l}$であり, 鏡映である.
$S_n$は互換で生成されていたので, $W$は$T_{k,l}$で生成される.
したがって, $W$は有限鏡映群である.


$S_n$は隣接互換$\tau_{1,2},\tau_{2,3},\ldots,\tau_{n-1,n},$
で生成されていたので,
$W$は$T_{1,2},T_{2,3},\ldots,T_{n-1,n}$という$n-1$個の行列で生成できることがわかる.


\begin{prop}
$\alpha=\ee_1+\cdots+\ee_n$とおく.
このとき, 任意の$A\in W$に対し,
$A\alpha=\alpha$となる.
したがって, $W$は本質的ではない.
\end{prop}
\begin{proof}
  $\sigma\in S_n$とする.
  このとき,
  \begin{align*}
    A_\sigma \alpha=\sum_{i=1}^{n}A_\sigma \ee_i=\sum_{i=1}^{n} \ee_{\sigma(i)}
  \end{align*}
  となるが, $\sigma$は$\Set{1,\ldots,n}$上の全単射であるので,
  \begin{align*}
    \sum_{i=1}^{n} \ee_{\sigma(i)}=\sum_{i=1}^{n} \ee_i=\alpha
  \end{align*}
  となる.
\end{proof}


\subsection{$A_1\times \cdots \times A_1$}
$W$を
\begin{align*}
 W= \Set{\diag(a_1,\ldots,a_n)|a_i\in \ZZ^\times}
\end{align*}
で定義する.
このとき, $W$は群である.
また,
$(a_1,\ldots,a_n)\in (\ZZ^\times)^n$に対し,
$f((a_1,\ldots,a_n))=\diag(a_1,\ldots,a_n)$
とすると,
$f$は$(\ZZ^\times)^n$から$W$への全単射である.
また, 準同型写像でもあるので,
$(\ZZ^\times)^n$と$W$は同型である.

また, $(-1,1,\ldots,1)$,\ldots,$(1,\ldots,1,-1)$で$ (\ZZ^\times)^n$は生成されていたので,
$F_1$,\ldots,$F_n$で$W$は生成される.
$F_k$は鏡映であったので,
$W$は有限鏡映群である.

\begin{prop}
  $W$は本質的である.
\end{prop}
\begin{proof}
  $\alpha\in V$とする.
  $A=\diag(-1,\ldots,-1)\in W$とすると,
  $A\alpha=-\alpha$である.
  $\alpha\neq \zzero$とすると,
  $0$ではない第$i$成分$\alpha_i$が存在する.
  このとき, $-\alpha_i\neq \alpha$であるので,
  $A\alpha\neq \alpha$である.
\end{proof}


\subsection{$B_n$}

$W$を
\begin{align*}
  W=\Set{\diag(\aaa)A_\sigma|\aaa
    \in (\ZZ^\times)^n, \sigma\in S_n}
\end{align*}
で定義する.
\begin{align*}
  A_\sigma \diag(a_1,\ldots,a_n)A_{\sigma}^{-1}\ee_{k}
&=  A_\sigma \diag(a_1,\ldots,a_n)A_{\sigma^{-1}}\ee_{k}\\
&=  A_\sigma \diag(a_1,\ldots,a_n)\ee_{\sigma^{-1}(k)}\\
&=  A_\sigma (a_{\sigma^{-1}(k)}\ee_{\sigma^{-1}(k)})\\  
&=  a_{\sigma^{-1}(k)}A_\sigma \ee_{\sigma^{-1}(k)}\\  
&=  a_{\sigma^{-1}(k)} \ee_{\sigma\sigma^{-1}(k)}\\  
&=  a_{\sigma^{-1}(k)} \ee_{k}\\  
\end{align*}
であるので,
\begin{align*}
  A_\sigma \diag(a_1,\ldots,a_n)A_{\sigma}^{-1}
  =\diag(a_{\sigma^{-1}(1)},\ldots,a_{\sigma^{-1}(n)})
\end{align*}
である.  つまり,
\begin{align*}
  A_\sigma \diag(a_1,\ldots,a_n)
  =\diag(a_{\sigma^{-1}(1)},\ldots,a_{\sigma^{-1}(n)})A_{\sigma}
\end{align*}
であることから,
\begin{align*}
  (\diag(b_1,\ldots,b_n)A_\tau)( \diag(a_1,\ldots,a_n)A_{\sigma})
  =\diag(b_1a_{\tau^{-1}(1)},\ldots,b_na_{\tau^{-1}(n)})A_{\tau\sigma} \in W
\end{align*}
となる.
よって, $W$は群である.
$\diag(\aaa)$も$A_\sigma$も鏡映の積としてかけるので,
$\diag(\aaa)A_\sigma$も鏡映の積としてかける.
つまり, $W$は有限鏡映群である.

\begin{prop}
  $W$は本質的である.
\end{prop}
\begin{proof}
$\diag(-1,\ldots,-1)\in W$であることからすぐわかる.
\end{proof}


\subsection{$D_n$}
$W$を
\begin{align*}
  W=\Set{ \diag(a_1,\ldots,a_n) A_{\sigma} | \sigma\in S_n , (a_1,\ldots,a_n)\in (\ZZ^{\times})^n, a_1\cdots a_n =1}
\end{align*}
とおく,
\begin{align*}
  (\diag(b_1,\ldots,b_n)A_\tau)( \diag(a_1,\ldots,a_n)A_{\sigma})
  =\diag(b_1a_{\tau^{-1}(1)},\ldots,b_na_{\tau^{-1}(n)})A_{\tau\sigma} \in W
\end{align*}
であるが, $b_1\cdots b_n=a_1\cdots a_n=1$ならば$b_1a_{\tau^{-1}(1)}\cdots b_na_{\tau^{-1}(n)}=1$
である. したがって,
$W$は群である.

$a_1\cdots a_n = 1$とすると,
$a_i=-1$となる$i$は偶数個である.
$2k$個の$i$で$a_i=-1$であるとし,
$\Set{i|a_i=-1}=\Set{i_1,\ldots,i_{2k}}$とする.
このとき, $\nu=\tau_{i_1i_2}\cdots \tau_{i_{2k-1}i_{2k}}$
とおくと,
\begin{align*}
F_{i_1,i_2}\cdots F_{i_{2k-1},i_{2k}}=\diag(a_1,\ldots,a_n)A_{\tau}
\end{align*}
である.  したがって$\sigma\in S_n$に対し,
\begin{align*}
  \diag(a_1,\ldots,a_n)A_{\sigma}
  &=\diag(a_1,\ldots,a_n)A_{\tau}A_{\tau}^{-1} A_{\sigma}\\
  &=F_{i_1,i_2}\cdots F_{i_{2k-1},i_{2k}}A_{\tau^{-1}} A_{\sigma}\\
  &=F_{i_1,i_2}\cdots F_{i_{2k-1},i_{2k}}A_{\tau^{-1}\sigma}
\end{align*}
とかける.
したがって, $W$は$\Set{F_{k,l}, T_{k,l}|k\neq l}$
で生成される.
$F_{k,l}$も$T_{k,l}$も鏡映であったから
$W$は有限鏡映群である.


\begin{prop}
$W$は本質的である.
\end{prop}
\begin{proof}
  $\alpha\in V\setminus\Set{\zzero}$とする.
  $\alpha\neq \zzero$なので,
  $0$ではない第$k$成分$\alpha_k$が存在する.
  \begin{align*}
    F_{k,l}T_{k,l}=F_kF_lT_{k,l}T_{k,l}=F_{k}F_{l}
  \end{align*}
  となるので,
  $F_{k}F_{l}\in W$である.
  $-\alpha_k\neq \alpha$であるので
  $F_{k}F_{l}\alpha \neq \alpha$である.
\end{proof}


\chapter{ルート系}
\section{ルート系の定義と例}
\begin{definition}
  $\Phi \subset V \setminus \Set{\zzero}$とする.
  以下の条件を満たすとき, $\Phi$が$V$上の(広義)ルート系であると呼ぶ:
  \begin{enumerate}
  \item $\numof{\Phi}<\infty$
  \item 任意の$\alpha \in \Phi$に対し次が成り立つ:
    \begin{enumerate}
    \item $\RR\alpha \cap \Phi = \Set{\alpha,-\alpha}$.
    \item $s_\alpha (\Phi) = \set{s_\alpha(\beta)|\beta\in \Phi}=\Phi$.
    \end{enumerate}
  \end{enumerate}
\end{definition}

$\Phi\subset V\setminus \Set{\zzero}$に対し,
$R(\Phi)=\Set{s_\alpha | \alpha \in \Phi}$
とし,
$W(\Phi)=\Braket{R(\Phi)}_{\ZZ}$とおく.
定義から$W(\Phi)$は鏡映群であるが, 有限群とは限らない.

\begin{lemma}
  \label{lem:nonesspart}
  $f\in W(\Phi)$に対し,
  $f_{(\RR\Phi)^{\perp}}=\id_{(\RR\Phi)^{\perp}}$.
\end{lemma}
\begin{proof}
  $\beta\in (\RR\Phi)^{\perp}$とする.
  $\alpha\in \Phi$に対し,
  $\Braket{\alpha,\beta}=0$であるので,
  $s_{\alpha}(\beta)=\beta$である.
  $f\in W(\Phi)$とすると,
  $f=s_{\alpha_1}\cdots s_{\alpha_l}$とかけるので,  
  $f(\beta)=\beta$である.
\end{proof}

\begin{prop}
ルート系$\Phi$に対し,
$W(\Phi)$は有限鏡映群である.
\end{prop}
\begin{proof}
  $W(\Phi)$が有限集合であることを示す.
  $S_\Phi$を$\Phi$上の全単射を集めた集合とする.
  このとき, $\numof{S}=\numof{\Phi}!$であり,
  有限集合である.
  $W'=\Set{f\restrictedto{\Phi}|f \in W}$
  とおく.
  $\alpha\in\Phi$に対し, $s\restrictedto{\Phi}=\Phi$であるので,
  \begin{align*}
    s_\alpha\restrictedto{\Phi}\colon
    \Phi &\to \Phi \\
    \beta&\mapsto s_\alpha(\beta)
  \end{align*}
  は全単射である
  $f\in W(\Phi)$とすると,
  $f=s_{\alpha_1}\cdots s_{\alpha_l}$とかけるので,
  $f|_\Phi$も
  $\Phi$上の全単射である.
  つまり, $W'\subset S_\Phi$であり, $W'$は有限集合である.
  ここで
  \begin{align*}
     \varphi\colon
     W(\Phi) &\to W' \\
     f&\mapsto f|_{\Phi}
  \end{align*}
  とおく.
  $\varphi$は単射であることを示す.
  $f,f'\in W$が$f\restrictedto{\Phi}=f'\restrictedto{\Phi}$を満たすとする.
  $V=\RR\Phi \oplus (\RR\Phi)^\perp$と(内部)直和分解できるが,
  \cref{lem:nonesspart}より,
  $\beta\in (\RR\Phi)^\perp$に対し,
  $f(\beta)=f'(\beta)=\beta$である.
  また仮定より, $\beta\in \RR(\Phi)$に対しても,
  $f(\beta)=f'(\beta)$が成り立つので,
  $f=f'$である.
\end{proof}

\begin{prop}
  $W$がを有限鏡映群とする.
  このとき, $W=W(\Phi)$となる
  ルート系$\Phi$が存在する.
\end{prop}
\begin{proof}
  \begin{align*}
    R&=\Set{ s\in W| \text{$s$は鏡映}}\\
    \Phi&=\Set{\alpha \in V | s_\alpha \in R, \|\alpha \|=1}
  \end{align*}
  とすればよい.
  $W$は鏡映群であるので,
  $W=\Braket{R}_{\ZZ}=W(\Phi)$である.
  $\numof{\Phi}< |W| <\infty$である.
  また, 定義から, $\alpha\in \Phi$に対し,
  $\RR\alpha \cap \Phi = \Set{\alpha,-\alpha}$が成り立つ.
  $\alpha\in \Phi$,
  $f\in W$とすると,
  $fs_\alpha f^{-1}=s_{f(\alpha)}$である.
  $fs_\alpha f^{-1}$は$W$の元であり, $s_{f(\alpha)}$は鏡映であるので,
  $s_{f(\alpha)}\in R$がわかる.
  したがって
  $f(\alpha)\in \Phi$である.
\end{proof}


\section{ルート系の例}
\subsection{$I_2(m)$}
\subsection{$A_{1}\times\cdots\times A_{1}$}
\subsection{$A_{n-1}$}
\subsection{$B_{n}$}
\subsection{$D_{n}$}
