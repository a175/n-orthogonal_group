% !TeX root =./x2.tex
% !TeX program = pdfLaTeX
\chapter{鏡映群の定義}
\section{鏡映の定義}
$\RR$は実数全体のなす集合とする.

特に断らない限り,
$V$は$n$次元ユークリッド空間を表すとする.
つまり,
\begin{align*}
  V&=\RR^n\\
  &=\Set{\begin{pmatrix}\alpha_1\\\vdots\\\alpha_n\end{pmatrix} | \alpha_i \in \RR}.
\end{align*}
とする. また, 内積は $\Braket{\bullet,\bullet}$で表すとする.
つまり,
\begin{align*}
\alpha=\begin{pmatrix}\alpha_1\\\vdots\\\alpha_n\end{pmatrix},\beta\begin{pmatrix}\beta_1\\\vdots\\\beta_n\end{pmatrix}\in V
\end{align*}
に対し,
\begin{align*}
\Braket{\alpha,\beta}=\sum_{i=1}^n \alpha_i \beta_i
\end{align*}
とする.
$\zzero$は$V$の零ベクトルを表し,
$\Set{\ee_1,\ldots,\ee_n}$は$V$の標準基底であるとする.
つまり,
\begin{align*}
 \zzero&=\begin{pmatrix}0\\\vdots\\0\end{pmatrix}\\
 \ee_1&=\begin{pmatrix}1\\0\\\vdots\\0\end{pmatrix}\\
 &\vdots\\
 \ee_n&=\begin{pmatrix}0\\\vdots\\0\\1\end{pmatrix}
\end{align*}


$\alpha \in V\setminus\Set{\zzero}$に対し,
\begin{align*}
  H_\alpha = \Set{\beta \in V | \Braket{\alpha,\beta}=0}
\end{align*}
とおき, $\alpha$を法ベクトルとする(中心的な)超平面と呼ぶ.
また,
\begin{align*}
  \RR\alpha = \Set{c \alpha | c\in \RR}
\end{align*}
とおく. $\RR\alpha$は原点を通り$\alpha$を方向ベクトルとする直線である.
このとき, $V$は,
\begin{align*}
  V =   H_\alpha \oplus   \RR\alpha
\end{align*}
と(内部)直和に分解される.  つまり, 次の2つを満たす:
\begin{align*}
  &V = \Set{ \beta + \gamma | \beta \in H_\alpha, \gamma \in \RR\alpha},\\
  &H_\alpha \cap   \RR\alpha =\Set{\zzero}.
\end{align*}

\begin{definition}
  $\alpha \in V\setminus\Set{\zzero}$とし,
  $f\colon V\to V$は線型写像であるとする.
  次の2つの条件を満たすとき,
  $f$が$\alpha$に関する鏡映であるという:
  \begin{enumerate}
  \item $f(\alpha)=-\alpha$.
  \item $\beta \in H_\alpha \implies f(\alpha)=\alpha$.    
  \end{enumerate}
  $f$が$\alpha$に関する鏡映であるような$\alpha\in V\setminus\Set{\zzero}$が存在するとき,
  $f$は鏡映であるという.
\end{definition}
$n$次正方行列は,
$\alpha\in V$に対し,
$A\alpha\in V$を対応させる$V$から$V$への線型写像であると思うことができる.
対応する線型写像が鏡映であるとき, その行列も鏡映と呼ぶことにする.
行列の言葉に定義を書き直すと以下のようになる.
\begin{definition}
  $\alpha \in V\setminus\Set{\zzero}$とし,
  $A$は$n$次正方行列であるとする.
  次の2つの条件を満たすとき,
  $A$が$\alpha$に関する鏡映であるという:
  \begin{enumerate}
  \item $A\alpha=-\alpha$.
  \item $\beta \in H_\alpha \implies A\alpha=\alpha$.    
  \end{enumerate}
\end{definition}

\begin{example}
  \begin{align*}
    P_0=
    \begin{pmatrix}
      1 & 0 \\
      0 & -1 
    \end{pmatrix}.
  \end{align*}
  とし,
  $\alpha=\ee_2$とおくと,
  $P_0$は$\alpha$に関する鏡映である.
実際, 
\begin{align*}
  P_0\alpha = 
      \begin{pmatrix}
      1 & 0 \\
      0 & -1 
      \end{pmatrix}
      \alpha
    =
      \begin{pmatrix}
      1 & 0 \\
      0 & -1 
    \end{pmatrix}
    \begin{pmatrix}
      0  \\
      1  
    \end{pmatrix}
    =
    \begin{pmatrix}
      0  \\
      -1  
    \end{pmatrix}
    =
    -
    \begin{pmatrix}
      0  \\
      1  
    \end{pmatrix}
    =
    -\alpha
\end{align*}
である.  また,
\begin{align*}
  H_\alpha = \Set{c \ee_1 | c\in \RR}
\end{align*}
であるが,
\begin{align*}
    \begin{pmatrix}
      1 & 0 \\
      0 & -1 
    \end{pmatrix}
    \begin{pmatrix}
      c  \\
      0  
    \end{pmatrix}
    =
    \begin{pmatrix}
      c  \\
      0  
    \end{pmatrix}
\end{align*}
である.
\end{example}

\begin{example}
  $E_n$で$n$次単位行列を表すことにする.
  つまり,
  \begin{align*}
    E_n=\diag(1,\ldots,1)
  \end{align*}
  である. $E_n$は鏡映ではない.
  $\alpha\in V$に対し,
  \begin{align*}
    E_n\alpha=-\alpha
  \end{align*}
  とすると,
  \begin{align*}
    \alpha=\zzero
  \end{align*}
  となり, $V\setminus\Set{\zzero}$の元では,
  $E_n\alpha=-\alpha$を満たすことがないので,
  $E_n$は鏡映ではない.
\end{example}

\begin{example}
  $P_0$は鏡映であったが,
  もっと一般に,
  実数$\theta$に対し,
\begin{align*}
  P_{2\theta}=
  \begin{pmatrix}
    \cos(2\theta) & \sin(2\theta) \\
    \sin(2\theta) & -\cos(2\theta) 
  \end{pmatrix},\\
  \alpha_\theta
  =
  \begin{pmatrix}
    \sin(\theta) \\
    \cos(\theta) 
  \end{pmatrix}
\end{align*}
とおけば, $P_{2\theta}$は$\alpha_\theta$に関する鏡映である.
実際,
\begin{align*}
  H_{\alpha_{\theta}} = \Set{c \alpha_{\theta+\frac{\pi}{2}}|c\in \RR} 
\end{align*}
であるので,
\begin{align*}
  P_{2\theta}\alpha_{\theta} &= - \alpha_{\theta},\\
  P_{2\theta}(c\alpha_{\theta+\frac{\pi}{2}}) &= c\alpha_{\theta+\frac{\pi}{2}}
\end{align*}
を示せばよいが, これらは直接, 計算することで確かめることができる.
\end{example}

\begin{example}
 実数$\theta$に対し,
  \begin{align*}
    R_\theta=
    \begin{pmatrix}
      \cos(\theta) & -\sin(\theta) \\
      \sin(\theta) & \cos(\theta)
    \end{pmatrix}.
  \end{align*}
 とおくと, これは鏡映ではない.
 \begin{align*}
    R_\theta \alpha = -\alpha
 \end{align*}
 とする. このとき,
 \begin{align*}
    (R_\theta + E_2)\alpha = \zzero
 \end{align*}
 である.
 \begin{align*}
   \det(R_\theta + E_2) &= 
    \begin{pmatrix}
      \cos(\theta) +1 & -\sin(\theta) \\
      \sin(\theta) & \cos(\theta) +1 
    \end{pmatrix}\\
    &=(\cos(\theta) +1)^2 + (\sin(\theta))^2\\
    &=2(1+\cos(\theta))
 \end{align*}
 である.
 したがって, $\cos(\theta)\neq 0$のときには,
 $R_\theta + E_2$は正則であるから, $\alpha=\zzero$のみが解となり,
 この場合は鏡映ではないことがわかる.
 また,  $\cos(\theta)= 0$のときには, $R_\theta=E_2$であり,
 この場合も鏡映ではない.
\end{example}

\begin{example}
  $P_0$は鏡映であった.
  $F_k$を$(k,k)$成分は$-1$, 他の対角成分は$1$であるような$n$次対角行列とする.
つまり,
  \begin{align*}
    F_k = \diag(1,\ldots,1,-1,1,\ldots,1)=E_n-2E_{k,k}
  \end{align*}
  である.
  $\alpha=\ee_k$とすると, $F_k$は$\alpha$ に関する鏡映である.
  $F_k \alpha = -\alpha$となることは, 直接計算すればわかる.
  また, $\beta\in V$に対し,
  \begin{align*}
    \Braket{\alpha,\beta}= \beta_k 
  \end{align*}
  であるので,
  \begin{align*}
    H_\alpha = \Set{
      \beta=
      \begin{pmatrix}
        \beta_1\\
        \vdots\\
        \beta_n
      \end{pmatrix}
      | \beta_k=0}
  \end{align*}
  である.  $\beta\in H_\alpha$に対し, 
  $F_k \beta = -\beta$となることは, 直接計算すればわかる.
\end{example}

\begin{example}
  $T_{k,l}$を$(k,k)$成分と$(l,l)$成分は$0$, 他の対角成分は$1$,
$(k,l)$成分と$(l,k)$成分は$1$, 他の成分は$0$
  であるような$n$次正方行列とする.
  つまり,
  \begin{align*}
    T_k =E_n-E_{k,k}-E_{l,l}+E_{l,k}+E_{k,l}
  \end{align*}
  である.
  $\alpha=\ee_k-\ee_l$とすると, $T_{k,l}$は$\alpha$ に関する鏡映である.
  実際,
  \begin{align*}
    T_{k,l} \alpha = T_{k,l}(e_k-e_l) = T_{k,l}e_k-T_{k,l}e_l = e_l - e_k = -(e_k - e_l) = -\alpha 
  \end{align*}
  である.  
  また, $\beta\in V$に対し,
  \begin{align*}
    \Braket{\alpha,\beta}= \beta_k - \beta_l
  \end{align*}
  であるので,
  \begin{align*}
    H_\alpha = \Set{
      \beta=
      \begin{pmatrix}
        \beta_1\\
        \vdots\\
        \beta_n
      \end{pmatrix}
      | \beta_k=\beta_l}
  \end{align*}
  である.
  $\beta\in H_\alpha$に対し, 
  $F_k \beta = -\beta$となることは, 直接計算すればわかる.
\end{example}



\begin{example}
  $k\neq l$に対し,
  $F_{k,l}=F_kF_lT_{k,l}$とおく.
  つまり$F_{k,l}$は$(k,k)$成分と$(l,l)$成分は$0$, 他の対角成分は$1$,
  $(k,l)$成分と$(l,k)$成分は$-1$, 他の成分は$0$であるような$n$次正方行列であり,
  \begin{align*}
    F_{k,l} = E_n-E_{k,k}-E_{l,l}-E_{k,l}-E_{l,k}
  \end{align*}
  である.
  $\alpha=\ee_k+\ee_l$とすると, $F_{k,l}$は$\alpha$ に関する鏡映である.
  実際,
  \begin{align*}
    F_{k,l} \alpha &= F_kF_lT_{k,l}(\ee_k+\ee_l)\\
    &= F_kF_l(\ee_k+\ee_l)\\
    &= F_kF_l\ee_k+F_kF_l\ee_l\\
    &= F_k\ee_k-F_k\ee_l\\
    &= -\ee_k-\ee_l\\
    &= -(\ee_k+\ee_l)\\
    &= -\alpha
  \end{align*}
  となる.
  また, $\beta\in V$に対し,
  \begin{align*}
    \Braket{\alpha,\beta}= -\beta_k-\beta_l 
  \end{align*}
  であるので,
  \begin{align*}
    H_\alpha = \Set{
      \beta=
      \begin{pmatrix}
        \beta_1\\
        \vdots\\
        \beta_n
      \end{pmatrix}
      | \beta_k=-\beta_l}
  \end{align*}
  である.  $\beta\in H_\alpha$に対し, 
  $F_k \beta = -\beta$となることは, 直接計算すればわかる.
\end{example}
