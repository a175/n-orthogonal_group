% !TeX root =./x2.tex
% !TeX program = pdfLaTeX
\chapter{鏡映群の定義}
\section{鏡映の定義}
$\RR$は実数全体のなす集合とする.

特に断らない限り,
$V$は$n$次元ユークリッド空間を表すとする.
つまり,
\begin{align*}
  V&=\RR^n\\
  &=\Set{\begin{pmatrix}\alpha_1\\\vdots\\\alpha_n\end{pmatrix} | \alpha_i \in \RR}.
\end{align*}
とする. また, 内積は $\Braket{\bullet,\bullet}$で表すとする.
つまり,
\begin{align*}
\alpha=\begin{pmatrix}\alpha_1\\\vdots\\\alpha_n\end{pmatrix},\beta\begin{pmatrix}\beta_1\\\vdots\\\beta_n\end{pmatrix}\in V
\end{align*}
に対し,
\begin{align*}
\Braket{\alpha,\beta}=\sum_{i=1}^n \alpha_i \beta_i
\end{align*}
とする.
$\zzero$は$V$の零ベクトルを表し,
$\Set{\ee_1,\ldots,\ee_n}$は$V$の標準基底であるとする.
つまり,
\begin{align*}
 \zzero&=\begin{pmatrix}0\\\vdots\\0\end{pmatrix}\\
 \ee_1&=\begin{pmatrix}1\\0\\\vdots\\0\end{pmatrix}\\
 &\vdots\\
 \ee_n&=\begin{pmatrix}0\\\vdots\\0\\1\end{pmatrix}
\end{align*}


$\alpha \in V\setminus\Set{\zzero}$に対し,
\begin{align*}
  H_\alpha = \Set{\beta \in V | \Braket{\alpha,\beta}=0}
\end{align*}
とおき, $\alpha$を法ベクトルとする(中心的な)超平面と呼ぶ.
また,
\begin{align*}
  \RR\alpha = \Set{c \alpha | c\in \RR}
\end{align*}
とおく. $\RR\alpha$は原点を通り$\alpha$を方向ベクトルとする直線である.
このとき, $V$は,
\begin{align*}
  V =   H_\alpha \oplus   \RR\alpha
\end{align*}
と(内部)直和に分解される.  つまり, 次の2つを満たす:
\begin{align*}
  &V = \Set{ \beta + \gamma | \beta \in H_\alpha, \gamma \in \RR\alpha},\\
  &H_\alpha \cap   \RR\alpha =\Set{\zzero}.
\end{align*}

\begin{definition}
  $\alpha \in V\setminus\Set{\zzero}$とし,
  $f\colon V\to V$は線型写像であるとする.
  次の2つの条件を満たすとき,
  $f$が$\alpha$に関する鏡映であるという:
  \begin{enumerate}
  \item $f(\alpha)=-\alpha$.
  \item $\beta \in H_\alpha \implies f(\alpha)=\alpha$.    
  \end{enumerate}
\end{definition}
$n$次正方行列は,
$\alpha\in V$に対し,
$A\alpha\in V$を対応させる$V$から$V$への線型写像であると思うことができる.

\begin{definition}
  $\alpha \in V\setminus\Set{\zzero}$とし,
  $A$は$n$次正方行列であるとする.
  次の2つの条件を満たすとき,
  $A$が$\alpha$に関する鏡映であるという:
  \begin{enumerate}
  \item $A\alpha=-\alpha$.
  \item $\beta \in H_\alpha \implies A\alpha=\alpha$.    
  \end{enumerate}
\end{definition}

\begin{example}
  \begin{align*}
    \begin{pmatrix}
      1 & 0 \\
      0 & -1 
    \end{pmatrix}.
  \end{align*}
は鏡映である.
\end{example}
